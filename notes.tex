\documentclass{book}         		                % The modifi
\usepackage{danielphysics}
\title{Plasma Theory and Diagnostics}
\author{Daniel Williams}

\begin{document} \maketitle

% \begin{abstract}
%   These notes were made during the Plasma Theory and Diagnostics
%   course taught in the School of Physics and Astronomy at the
%   University of Glasgow during the 2013 -- 2014 academic session.
% \end{abstract}

\tableofcontents

\chapter{The Mechanics of Plasma}
\label{cha:mechanics-plasma}

\section{Particle Orbit Theory}
\label{sec:part-orbit-theory}

At a first glance, treating the individual particles of a plasma seems
an inefficient way to develop understanding of plasmas, but it is in
fact the essential underlying physics of the entire theory of plasmas.
Orbit theory is only valid with high-energy, low density plasmas with
infrequent collisions. We also need symmetric external fields. \\
In this section it is assumed that radiative effects are negligible,
and that particle energies are low enough that we need only consider
the non-relativistic Lorentz equation, for a particle of mass $m_j$,
and charge $q_j$ at a position $\vec{r}$ in an electric field
$\vec{E}(\vec{r}, t)$, and magnetic field $\vec{B}(\vec{r}, t)$,
\begin{equation}
  \label{eq:lorentzeq}
  m_j \dv[2]{\vec{r}_j}{t} = q_j \qty[ \vec{E}(\vec{r},t) + \dot{\vec{r}_j} \times \vec{B}(\vec{r}, t) ]
\end{equation}
Whenever charge distributions or current densities are signifigant a
statistical or fluid approach will be required.

% Consider an ensemble of plasma particles, $i = 1, \dots, N$, then we
% can assemble a Hamiltonian for the system,
% \[ H = (\vec{r}_1, \dots, \vec{r}_N; \vec{p}_1, \dots, \vec{p}_N) \]
% so we clearly have $6N$ degrees of freedom. Thus, solving the
% Hamiltonian for the whole system is impractical for almost every
% conceivable system. There may be systems where there are very small
% number of particles, $M \ll N$, so that we can solve the system
% \[ \dv{\vec{p}}{t} = q \qty[ \vec{E} + \vec{v} \times \vec{B}] \]
% using PDE solvers. This is a \emph{test particle} description. Another
% case where it may be practical is when the mean free path of the gas
% is much smaller than the size of a perturbation, for example, sound
% waves in air, or a high density magnetohydrodynamic description. The
% kinetic description extends the treatment of plasma to a full
% statistical description to avoid a number of fundamental ambiguities
% in the MHD description.

\subsection{Constant Homogeneous Magnetic Field}
\label{sec:const-homog-magn}

Let us start with a static description, $\vec{E} = 0, \vec{B}(\vec{r},
t) = B \neq 0$.  Taking the direction of $\vec{B}$ to define the
$z$-axis, so $\vec{B} = (0,0,B)$, we have the scalar product of
$\hat{z}$ and the Lorentz equation, equation (\ref{eq:lorentzeq}) giving 
\begin{equation}
  \label{eq:no-z-accel}
  \ddot{z} = 0
\end{equation}
implying that $\dot{z} = v_{\parallel}$ is constant. Likewise, $m
\ddot{\vec{r}} \cdot \dot{\vec{r}} = 0$, so $\half m \dot{\vec{r}}^2$ is
constant.  We can now find, by taking dot products of the unit vectors
with equation (\ref{eq:lorentzeq}), that
\begin{align*}
  \dv{\vec{v}_z}{t}                & = 0                                                            \\
  \dv{\vec{v}_x}{t}                & = -\frac{q B \vec{v}_y}{m}                                     \\
  \dv{\vec{v}_y}{t}                & = - \frac{q B \vec{v}_x}{m}
\end{align*}
We now have a number of conditions which simlify the system, so, from
equation (\ref{eq:lorentzeq}),
\begin{equation} 
\dv{\vec{v}}{t} = \frac{q}{m} \vec{v} \times \vec{B} 
\end{equation}
and since $\vec{v}_z = {\rm const}$,
\[ \vec{z} - \vec{z}_0 = \vec{v}_{z0}t \]
The $x$ and $y$ motions are more complicated;
\begin{align*}
  \dv[2]{\vec{v}_x}{t}             & = \frac{qB}{m} \dv{v_y}{t}                                                                      \\
                                   & = - \qty( \frac{q B}{m})^2 v_x                                                                  \\
  \ddot{\vec{v}}_x                 & = - \omega_{\rm c}^2 \vec{v}_x                                                                  \\
  v_x                              & = v_{x0} \exp( \pm i \omega_{\rm c} t + i \phi_x)                                               \\
  \Re(v_x)                         & = v_{x0} \cos( \omega_{\rm c} t)
\end{align*}
\begin{align*}
  \dv[2]{v_y}{t}                   & = - \qty(\frac{qB}{m})^2 v_y                                                                    \\
  v_y                              & = \frac{m}{qB} \dv{v_x}{t}                                                                      \\
                                   & = - \frac{m}{qB} v_{x0} \sin( \omega_{\rm c} t) \omega_{\rm c}                                  \\
                                   & = \mp v_{x0} \sin( \omega_{\rm c} t )                                                           \\
                     \end{align*}

Before proceeding, we'll make the definition of the \emph{Larmour Radius},
\begin{definition}[Larmour Radius]
  The Larmour radius, $r_{\rm L}$ (also gyroradius, or cyclotron radius) is the
  radius of the circular motion of a charged particle in a uniform
  magnetic field.
\[ r_{\rm L} = \frac{m v_{\perp}}{|q|B} \]
\end{definition}

We can now find the full equations of motion by considering
\begin{subequations}
\begin{align}  \nonumber
                     v_x^2 + v_y^2 & = v_{x0}^2 \qty( \cos[2](\omega_{\rm c} t) + \sin[2]({\omega_{\rm c}t}) ) = v_{x0}^2 = v_{\perp}^2 \\ \nonumber
  \vec{v}                          & = (v_{\parallel}, v_{\perp})                                                              \\ 
  x - x_0                          & = \frac{v_{\perp}}{\omega_{\rm c}} \sin(\omega_{\rm c}t)     = r_{\rm L} \sin(\omega_{\rm c} t)  \\
  y - y_0                          & = \frac{\mp v_{\perp}}{\omega_{\rm c}} \cos(\omega_{\rm c}t)   =\mp r_{\rm L} \cos(\omega_{\rm c} t)  \\ 
  z - z_0 &= v_{z0} t = v_{\parallel} t \\
\nonumber
\text{since }  \frac{v_{\perp}}{\omega_{\rm c}}    & = \frac{v_{\perp}m}{|q|B} \equiv
  r_L
\end{align}
\end{subequations}

\subsection{Constant Homogeneous Magnetic and Electric Fields}
\label{sec:const-homog-magn-et-elec}

Now let's elaborate to a situation which involves an electric
field. We now have, from equation (\ref{eq:lorentzeq}),
\begin{equation} 
m \dv{\vec{v}}{t} = q\vec{E} + q \vec{v} \times \vec{B}
\end{equation}
and take the inner product of this with the velocity,

\begin{align}
\nonumber m\vec{v}\cdot \dv{\vec{v}}{t}             & = q \vec{v} \cdot \vec{E} + 0                 \\
\nonumber \dv{t} \qty(\frac{mv^2}{2})          & = q \vec{v} \cdot \vec{E}                     \\
\nonumber                                      & = - q \vec{v} \nabla \phi                     \\
\nonumber                                      & = -q \dv{\vec{r}}{t} \cdot \dv{\phi}{\vec{r}} \\
\nonumber                                      & = -q \dv{\phi}{t}                             \\
\nonumber  \dv{t} \qty[ \frac{mv^2}{2}+q \phi] & = 0 
\end{align}

Now we arrange our coordinates such that
$\vec{B}=(0,0, B_z)$, and $\vec{E} = (E_x, 0, E_z)$, so
\begin{subequations}
\begin{align}
\dv{v_z}{t} & = \frac{q}{m} E_z                                              \\
\dv{v_x}{t} & = \frac{q}{m} E_x + \frac{qB_z}{m} v_y                         \\
\dv{v_y}{t} & = - \frac{qB}{m} v_x 
\end{align}
\end{subequations}
The gyrocentre will shift, but to understand how we must solve these equations.
To do this we take a similar approach to the last lecture.
\begin{align*}
\ddot{v}_x  & = 0 + \frac{qB}{m} \dot{v}_y                                   \\
            & = - \qty(\frac{qB}{m})^2 v_x                                   \\
            & = - \omega_{\rm c}^2 v_x                                       \\
\ddot{v}_y  & = -\omega_{\rm c} \dot{v}_x                                    \\
            & = - \frac{qB}{m} \qty( \frac{q}{m} E_x + \frac{qB}{m} v_y )    \\
            & = -\omega_{\rm c}^2 \qty( v_y + \frac{q}{m} \frac{m}{qB} E_x ) \\
            & = - \omega_{\rm c}^2 \qty( v_y + \frac{E_x}{B} )               \\
\dv[2]{t}\qty( v_y + \frac{E_x}{B} ) &= - \omega_{\rm c}^2 \qty( v_y + \frac{E_x}{B} ) 
\end{align*}
Then, at $t=0$, $v_x = v_{x0}$, and $v_y=0$,
\begin{subequations}
\begin{align}
v_x &= v_{x0} \cos(\omega_{\rm c} t) \\
v_y &= \mp v_{x0} \sin(\omega_{\rm c}) - \frac{E_x}{B} 
\end{align}

Let's take an alternative approach here, and work entirely with vectors again, so first,
\begin{align}
\vec{B} \times m \dv{\vec{v}}{t} &= \vec{B} \times \qty(q \vec{E} + q \vec{v} \times \vec{B})  \nonumber \\
q \vec{B} \times \vec{E} + q \vec{B} \times \qty( \vec{v} \times \vec{B}) &= 0  \nonumber \\
q \vec{B} \times \vec{E} + q \qty(\vec{v} \cdot B^2 - \vec{B} \qty( \vec{v} \cdot \vec{B})) &= 0  \nonumber\\
\vec{v}_{\rm ge} B^2 &= \vec{E} \times \vec{B}  \nonumber\\
\vec{v}_{\rm ge} &= \frac{\vec{E} \times \vec{B}}{B^2} 
\end{align}
\end{subequations}
with $v_{\rm ge}$ the velocity of the guiding centre.

\subsection{Drift in an external field}
\label{sec:drift-an-external}


As a result of this drift, the movement of a particle in this situation can be described by
\begin{align*}
  \label{eq:1}
  \vec{v} &= \vec{u} + \vec{v}_{\rm ge} \\
m \dv{\vec{v}}{t} &= q\vec{E} + q \vec{v} \times \vec{B} \\
\dv{\vec{v}}{t} &= \dv{\vec{u}}{t} + \dv{\vec{v}_{\rm ge}}{t} \\
m \dv{\vec{v}}{t} &= m \dv{\vec{u}}{t} \\ &= q \vec{E} + q \qty( \vec{u} + \vec{v}_{\rm ge}) \times \vec{B}\\
&= q \vec{E} + q \vec{u} \times \vec{B} + q \vec{v}_{\rm ge} \times \vec{B} \\
m \dv{u_{\parallel}}{t} &= q E_{\parallel} \\
m \dv{v_{\rm ge}}{t} &= q \vec{u}_{\perp} \times \vec{B} \\
\vec{u} &= u_{\parallel} \frac{\vec{B}}{|B|} + \vec{u}_{\perp} + \vec{v}_{\rm ge} \\
m \dv{\vec{v}}{t} &= \vec{F}_{\rm ext} + q \vec{v} \times \vec{B} \\
\vec{v}_{\rm F} &= \frac{1}{q} \frac{\vec{F} \times \vec{B}}{B^2} \\
\vec{v}_{\rm g} &= \frac{m}{q} \frac{\vec{g} \times \vec{B}}{B^2}
\end{align*}

Now, suppose we have a non-uniform field which varies weakly, that is,
\[ \nabla \cdot \vec{B} \sim \frac{B}{L} \] for $L \gg r_{\rm L}$.
Again, we set up our coordinates so $\vec{B} = (0,0, B)$, then we have
\begin{align*}
  \langle \vec{F} \rangle &= \frac{1}{T} \int_0^T F(t) \dd{t} \\
F &= q \vec{v} \times \vec{B} = \hat{\imath} v_y B - \hat{\jmath} v_x B + \hat{k} \cdot 0 \\
\vec{B}(\vec{R}_{\rm v}+\vec{r}) &= \vec{B}_0 + ( \vec{r} \nabla ) \vec{B} + \cdots \\
B &= B_0 + (\vec{r} \nabla) B \\
B &= B_0 + y \cdot \pdv{y} B \\
F_x &= q v_y B \\
&= q v_y (B_{0z} +y \pdv{y}B) \\
F_y &= -q v_x B \\ &= q v_x ( B_{0z} + y \pdv{B}{y} )
\end{align*}
Now, using the results from earlier, for the unperturbed situations,
\begin{align*}
  F_x &= \mp q v_{x0} \sin( \omega_{\rm c} t ) \qty( B_{0z} \pm \frac{v_{x0}}{\omega_{\rm c}} \cos(\omega_{\rm c} t) \pdv{B}{y} ) \\
F_y &= - q v_{x0} \cos( \omega_{\rm c} t) \qty( B_{0z} \pm \frac{v_{x0}}{\omega_{\rm c}} \cos(\omega_{\rm c} t) \pdv{B_z}{y} )
\end{align*}
Then, knowing,
\begin{align*}
  \langle \sin(\omega_{\rm c}t) \rangle &= 0 \\
\langle \cos(\omega_{\rm c}t) \rangle &= 0 \\
\langle \sin(\omega_{\rm c} t) \cos( \omega_{\rm c} t) \rangle &= 0 \\
\langle \cos(\omega_{\rm c} t) \cos( \omega_{\rm c} t) \rangle &= \half 
\end{align*}
so
\begin{align*}
  \langle F_x \rangle &= 0 \\
\langle F_y \rangle &= -q \frac{v_{x0}^2}{\omega_{\rm c}} \langle \cos[2](\omega_{\rm c}t) \rangle \pdv{B}{y} \\
&= \mp q \frac{v_{\perp}^2}{2 \omega_{\rm c}} \pdv{B}{y} \\
\langle F \rangle &= \mp q \frac{v_{\perp}^2}{2 \omega_{\rm c}} \nabla |\vec{B}| \\
v_{\rm ge} &= \frac{1}{q} \mp q \frac{v_{\perp}^2}{2 \omega_{\rm c}} \frac{\nabla |\vec{B}| \times \vec{B}}{B^2} \\
&= \pm \frac{v_{\perp}^2}{2 \omega_{\rm c}} \frac{\vec{B} \times \nabla|\vec{B}|}{B^2}
\end{align*}


\subsection{Inhomogeneous Magnetic Fields}
\label{sec:inhom-magn-fields}

In practice magnetic fields are rarely uniform, but are generally
space- and time-dependent. Normally this would lead to a numerical
treatment being required, but there are cases where it is possible to
calculate the variation analytically, by assuming the inhomogeneity to
be small.

First, consider the case when a particle moves only parallel to the
magnetic field lines. The central force experienced by the particle
will be
\begin{equation}
  \label{eq:centralforceinhommag}
  \vec{F}_{\rm cent} = \frac{m v_{\parallel}^2}{R_{\rm c}} \hat{R}_{\rm c}
\end{equation}
with $R_{\rm c}$ being the radius of curvature that the particle is
moving on. Now, we introduce the quantity
\[ \hat{B} \cdot \nabla := \pdv{S} \] as the directional derivative
along a field line, which is the rate of change of the magnetic field
along the direction $\hat{B}$, so
\begin{align*}
  \frac{\vec{R}_{\rm c}}{R_{\rm c}^2} &= - \pdv{S} \qty(
  \frac{\vec{B}}{B} ) \\&= - \frac{1}{B} \pdv{\vec{B}}{S} + \frac{\vec{B}}{B^2} \pdv{B}{S} \\
&= - \frac{1}{B^2} \qty( \vec{B} \cdot \nabla) \vec{B} + \frac{\vec{B}}{B^2} \pdv{B}{S} \\
\end{align*}
then, returning to the force equation,
(\ref{eq:centralforceinhommag}), and, since the particle is moving
along lines of constant field strength, so that $\pdv{B}{S}=0,$
\begin{equation}
  \label{eq:centforceinhommag2}
  \vec{F}_{\rm cent} = - \frac{mv_{\parallel}^2}{B^2} (\vec{B} \cdot \nabla) \vec{B}
\end{equation}
and the curvature drift velocity is then 
\begin{equation}
\begin{eq:curvedriftvelinhommag}
\vec{v}_{\rm D} = \frac{1}{q} \frac{\vec{F}_{\rm cent} \times \vec{B}}{B^2} = \frac{m v_{\parallel}^2}{q B^4} \qty( \vec{B} \times (\vec{B} \cdot \nabla) \vec{B} )
\end{equation}
and
\begin{equation}
  \label{eq:curvedriftradinhom}
\vec{v}_{\rm R} = \frac{m v_{\parallel}^2}{q B^2} \frac{\vec{R}_{\rm c} \times \vec{B}}{R_{\rm c}^2}
\end{equation}
Now, in reality there is a gradient ($\nabla \vec{B}$) drift which
accompanies the curvature drift, as $\vec{B}$ must decrease with
radius. This is because, in a vacuum we require $\nabla \times \vec{B}
= 0$ (law of conservation of energy) and $\nabla \cdot \vec{B} = 0$
(Gauss's Law). Expressing the problem in cylindrical coorinates it is
trivial to see that $\nabla \times \vec{B}$ can only have an
$z$-component. Now,
\[ \vec{B} = (0, B_{\theta}, 0)\]
so,
\[ ( \nabla \times \vec{B} )_z = \frac{1}{r} \pdv{r} (r B_{\theta}) =
0 \] and so $B_{\theta}\sim\frac{1}{r}$.  Hence, $B = B_{\theta} \sim
\frac{1}{R_{\rm c}}$, and, $\frac{\nabla B}{B} = - \frac{\vec{R}_{\rm
    c}}{R_{\rm c}^2}$. Using the gradient drift expression from earlier, 
\begin{equation}
  \label{eq:2}
  \vec{v}_{\nabla B} = \pm \frac{v_{\perp}^2}{\alpha \omega_{\rm c}} \frac{\vec{B} \times \nabla B}{B^2}
\end{equation}
and then substituting the $\nabla B$ due to the curvature, 
\begin{equation}
  \label{eq:3}
  \vec{v}_{\nabla B} = \pm \frac{v_{\perp}^2}{2 \omega_{\rm c}} \frac{\vec{R}_{\rm c} \times \vec{B}}{R_{\rm c}^2 B} = \half \frac{m v_{\perp}^2}{q} \frac{\vec{R}_{\rm c} \times \vec{B}}{R_{\rm c}^2 B^2}
\end{equation}
Then, the combined shift is
\begin{equation}
  \label{eq:4}
  \vec{v}_{\rm D} = \vec{v}_{R} + \vec{v}_{\nabla R} = \frac{m}{q} \frac{\vec{R}_{\rm c} \times \vec{B}}{R_{\rm c}^2 B^2} \qty( v_{\parallel}^2 + \half v_{\perp}^2 )
\end{equation}
so the total drift due to a non-uniform field is perpendicular to both
$R_{\rm c}$ and $\vec{B}$, and so in dipole fields we have drift which
is perpendicular to both $R_{\rm c}$ and $\vec{B}$, and the drift is
charge-dependent, so there is a current, which is known as the
\emph{ring current}.

\subsection{Magnetic Mirroring}
\label{sec:magnetic-mirroring}

Consider a non-uniform magnetic field, primarily in the $z$-direction,
which has a magnitude which varies in the $z$-direction. Let the field
be axisymmetric, such that $B_{\theta}=0$, and $\pdv{\theta}\cdot
B=0$. Since the field lines converge and diverge, $B_r \neq 0$.

\begin{figure}
  \centering
  \begin{tikzpicture}[scale=0.8]

\foreach \y in {1,2,3}{
	\pgfmathsetmacro{\yp}{\y-1}
	\draw [accent-blue, <-] (7,-\yp) .. controls (3,-\yp) and (2,-\y) .. (-1,-\y);
	\draw [accent-blue, <-] (7,\yp) .. controls (3,\yp) and (2,\y) .. (-1,\y);
}
\draw [ultra thick, ->] (-2,0) -- (8,0) node [right] {$z$};
\draw [thick, ->] (-1,0) -- (-1,1) node [midway, left] {$r$};
\end{tikzpicture}
  \caption{A non-uniform magnetic field, varying in the $z$-direction.}
  \label{fig:magneticmirror1}
\end{figure}

From $\nabla \cdot \vec{B} = 0$ we have 
\[ \frac{1}{r} \pdv{r} (r B_r) + \pdv{B_z}{z} = 0 \] If $\pdv{B_z}{z}$
at $r=0$ is given, and doesn't change much with $r$ we have the
relation
\[ r B_r = - \int_0^r r \pdv{B_z}{z} \dd{r} \approx \half r^2
\pdv{B_z}{z} \eval_{r=0} \] The variation of $B$ with $r$ causes a
$\nabla B$ drift of the guiding centre about the axis of symmetry, but
there is no radial $\nabla B$ drift, as $\pdv{B}{\theta}=0$. The Lorentz force is then
\begin{equation}
  \label{eq:mirroringlorentz}
  \vec{F} = q
  \begin{bmatrix}
    v_{\theta} B_z - v_z B_{\theta} \\ - v_r B_z + v_z B_r \\ v_r B_{\theta} - v_{\theta} B_r
  \end{bmatrix}
\end{equation}
If $B_{\theta}=0$ two terms are equal to zero, and as $r \to 0$, $B_r
\to 0$, since $B_r$ vanishes on the axis. When it doesn't vanish the
azimuthal force leads to a drift in the radial direction. This drift
makes the guiding centres follow the magnetic field lines. Consider
the $z$-component of (\ref{eq:mirroringlorentz}),
\[ F_z = -q v_{\theta} B_r = \half q v_{\theta} r \pdv{B_z}{z} \]
averaging over a gyration,
\[ \ev{F_z} = \mp \half q v_{\perp} r_{\rm L} \pdv{B_z}{z} = \mp \half
q \frac{v_{\perp}}{\omega_{\rm c}} \pdv{B_z}{z} \]
Now, making a definition,
\begin{definition}[Magnetic Moment]
\[ \mu := \half \frac{m v_{\perp}^2}{B} \]  
\end{definition}
The force can then be written
\begin{equation}
  \label{eq:5}
  \vec{F}_{\parallel} = - \mu \pdv{B}{S} = - \mu \nu_{\parallel} B
\end{equation}
Now we can consider the component of the equation of motion along
$\vec{B}$, 
\[ m \dv{v_{\parallel}}{t} = - \mu \pdv{B}{S} \] then multiplying by
$v_{\parallel}$,
\begin{align*}
  m v_{\parallel} \pdv{v_{\parallel}}{t} &= - \mu \pdv{B}{S} v_{\parallel} \\ &= - \mu \pdv{B}{S} \cdot \dv{S}{t} \\ &= -\mu \dv{B}{t}
\end{align*}
Here $\dv{B}{t}$ is the variation of $B$ as seen by the particle;
$\vec{B}$ itself is constant. Since $\frac{mv^2}{2}$ is constant, due
to conservation of energy, we have
\[ \dv{t} \frac{mv^2}{2} = \dv{t} \qty( \frac{mv_{\parallel}^2}{2} +
\frac{m v_{\perp}^2}{2} ) = \dv{t} \qty( \half mv_{\perp}^2 + \mu
B)=0 \]
since 
\[ m v_{\parallel} \dv{v_{\parallel}}{t} = \dv{t} \qty( \frac{mv^2}{2} ) = - \mu \dv{B}{t} \]
so
\[ - \mu \dv{B}{t} + \dv{t} \qty( \mu B ) = 0 \Leftrightarrow
\dv{\mu}{t} = 0 \] and thus the magnetic moment is a conserved
quantity, and is invariant in time. This invariance allows a plasma to
be confined through magnetic mirrors.

As the particle moves from the weak field region into the strong-field
region, $v_\perp$ must increase in order that $\mu$ stay
constant. Since $v_\perp^2 + v_\parallel^2$ is constant, it follows
that $v_\parallel$ must decrease, eventually to $0$. Eventually the
particle will be reflected back towards the weak-field region. We can
then associate a force, $F_{\parallel}$, or the \emph{mirror force}
with this action, and the plasma is magnetically trapped between two
magnetic mirrors. \\
The trapping is not, however, perfect. If a particle has $v_{\perp}=0$
it will have no $\mu$, and so will not feel a mirror force. As a
result, at small $\frac{v_{\perp}}{v_{\parallel}}$ we expect that there
will be some escape of particles.\\
Consider a particle with $v_{\parallel,0}$ and$v_{\perp,0}$ initially
in the region $B_{\rm min}$. By conservation of $\mu$,
\[ \frac{1}{2} \frac{m v_{\perp,0}^2}{B_{\rm min}} = \half \frac{m
  v_{\perp}^{\prime 2}}{B_{\rm max}} \]
Then, by conservation of $v_{\perp}^2 + v_{\parallel}^2$,
\[v_0^2 = v_{\perp,0}^2 + v_{\parallel,0}^2 = v_{\perp}^{\prime 2} + v_{\parallel}^{\prime 2} \]
and so
\[ \frac{B_{\rm min}}{B_{\rm max}} = \frac{v_{\perp,
    0}^{2}}{v_{\perp,0}^{\prime 2}} = \frac{v_{\perp,0}}{v_0^2} =
\sin[2](\theta) \] so
\[ \sin(\theta) := \frac{v_{\perp}}{v_0} \] We can now use $\theta$ to
describe whether the particle escapes from, or is trapped by, the
magnetic field.

\begin{equation}
  \label{eq:6}
  \sin[2](\theta) = 
  \begin{cases}
    < \frac{B_{\rm min}}{B_{\rm max}} & \text{escape} \\
> \frac{B_{\rm min}}{B_{\rm max}} & \text{trapped}
  \end{cases}
\end{equation}


\subsection{Adiabatic Invariants}
\label{sec:adiabatic-invariants}

In a classical system which experiences a periodic motion, the action
integral taken over one period of the motion will be constant. That is
\[ \oint \vec{p} \dd{\vec{q}} = \text{constant} \] If a slow change
occurs in the system, such that the motion is (just) non-periodic, the
constant of the motion does not change, and is called adiabatically
invariant. The slow change is qualified by a change takes longer than
one period of the underlying motion, so that the action is
well-defined (although the integral is strictly no longer a closed
loop integral).

\subsubsection{The First Adiabatic Invariant}
\label{sec:first-adiab-invar}

The first adaibatic invariant of a plasma involves the Larmour gyration, so
\begin{equation}
  \label{eq:7}
  \oint \vec{p} \dd{\vec{q}} = \oint m v_{\perp} r_{\rm L} \dd{\phi} = 2 \pi m v_{\perp} r_{\rm L} = 4 \pi \frac{m}{\abs{q}} \mu 
\end{equation}
The magnetic moment, $\mu$, will be conserved when the variation time,
$\Delta t$ of the magnetic field, $B$, is long, i.e.\ $\Delta t
\omega_{\rm c} \gg 1$. Otherwise the magnetic moment is not conserved.

\subsubsection{The Second Adiabatic Invariant}
\label{sec:second-adiab-invar}

The second adiabatic invariant involves the periodic oscillation of
plasma particles between magnetic mirrors. This time
\begin{equation}
  \label{eq:8}
  \oint m v_{\parallel} \dd{S} = \text{constant}
\end{equation}
Since the guiding centre drifts across field lines, however, the
motion is not exactly periodic, and so the motion is adiabatically
invariant. This is also longitudinally invariant $J$, defined as the
half-cycle between the two mirror points, $a$ and $b$, with
\[ J= \int_a^b v_{\parallel} \dd{S} \] with $S$ a path along a field
line.  $J$-invariance is violated in \emph{transit-time magnetic
  pumping}, a method for heating a plasma. This is done by moving $a$
and $b$ over time to increase the $v_{\parallel}$ as the particles
approach the mirror points.

\chapter{The Bulk Properties of Plasma}
\label{cha:bulk-prop-plasma}

Plasma is the ``fourth state of matter''. It refers to a state
containing enough free charges for its dynamics to be dominated by
long-range Coloumb forces, rather than shorter-range binary
collisions\footnote{Air generally has around $10^{9} {\rm m}^{-3}$
  ions and electrons which are caused by background sources and
  friction. These can be harnassed as seeds for further ionisation.}.
The presence of charge carriers in the matter cause a plasma to
interact strongly with electromagnetic fields.

\section{Producing a Plasma}
\label{sec:production}

There are a number of approaches to producing a plasma in the lab:
\begin{enumerate}
\item \textbf{photoionisation}---for this we need photons with
  sufficient energy to remove electrons from the neutral species,
  e.g. 13.6 eV for Hydrogen, 24.6 eV for Helium, and 15.6 eV for
  molecular nitrogen. All of these energies lie within the ultraviolet
  region of the electromagnetic spectrum. The existence of long-lived,
  metastable states can help with ionisation processes (He has two, at
  19.8 eV (with a half-life of 700s), and 20.61 eV; ${\rm H}_{2}$ at
  11.75 eV ($10^{-3} {\rm s}$), and H at 0.52 eV (1 month).)
\item \textbf{electron impact}---for this we accelerate any free
  electrons (for example, the seed electrons) in an electric field
  until they reach the ionising threshold. This process makes use of
  the Townsend process, where $N$ electrons move along the $x$-axis in
  the presence of a uniform electric field, $E$; then $\delta N$
  electrons are produced by electron-impact ionisation in a distance
  $\dif x$, according to
  \begin{equation}
    \label{eq:townsend}
    \delta N = \alpha_{{\rm T}} N \dif{x}
  \end{equation}
  with $\alpha_{{\rm T}}$ being the ionisation co-efficient.
  Thus,
  \begin{equation}
    \label{eq:townsendint}
    N(x) = N_0 \exp(\alpha_{\rm T} x)
  \end{equation}
\end{enumerate}
For this lecture course, we shall assume that the plasma is 100 \% ionised, i.e. there shall be no plasma-neutral interaction.\\

\section{Basic Physics of Plasmas}
\label{sec:basic-phys-plasm}

\subsection{Electrical Neutrality and the Debye Length}
\label{sec:neutralanddebye}

Suppose we have a gas of electrons and protons, i.e. a Hydrogen
plasma, at a temperature, $T$. Consider the situation where a random
fluctuation of the electron population exposes some positive
particles, thus an unbalanced positive charge. Exposing an unbalanced
positve charge will cause a net movement of negative charge (in the
form of electrons) to move towards the positive charge.  What is the
associated scale length for this process; can this be made consistent
with thermodynamics?
\begin{center}
\input{figures/onedmodel.pgf}
\end{center}
The average kinetic energy of the electrons is $E = \frac{1}{2} k_{B
}T$. The fluctuation in the electron denisty leaves behind an
unbalanced charge, and therefore an associated electric potential
$\phi$. Poisson's equation says
\[ \nabla^2 \phi = - \frac{ne}{\epsilon_{0}} \]
with $n$ being the number density. So in one dimension, 
\[ \frac{\difp{2} {\phi}}{\dif x^2} = - \frac{ne}{\epsilon_0} \]
which has a solution
\[ \phi(x) = {\rm const} - \frac{nex^2}{2 \epsilon_0} \] Note that
symmetry rules out a linear term in $x$.  Then we have boundary
conditions: $\phi(x=\pm d) = 0$.  Applying these boundary consitions
means that \[ \frac{ned^2}{2\epsilon_0} = {\rm const} \]

\begin{center}
\begin{tikzpicture}
\pgfplotsset{
    every axis/.append style={
        %scale only axis,
        width=0.95*\linewidth,
        height=1.5in,
    },
    /tikz/every picture/.append style={
        trim axis left,
        trim axis right,
    }
}
  \begin{axis}
    \addplot [mark=none, muted-orange, domain=-3:3, ultra thick] {x^2
      -9}; \addplot [mark=none, muted-green, domain=-5:-3, ultra
    thick] {0}; \addplot [mark=none, muted-green, domain=3:5, ultra
    thick] {0};
  \end{axis}
\end{tikzpicture}
\end{center}

Now recall that we want to create this region via a small thermal
fluctuation; therefore we need $\frac{1}{2} k_{B}T$ to be the maximum
energy of the electrons, and to be the maximum potential energy of the
well (otherwise the electrons can't escape.)  Hence
$\frac{ned^2}{2\epsilon_0} e = \frac{1}{2} k_BT$, and so 
\begin{equation}
d =\left[ \frac{\epsilon_0 k_BT}{ne^2} \right]^{\frac{1}{2}} = \lambda_{\rm D}
\end{equation}
which is the Debye length, which is conventially denoted
$\lambda_{{\rm D}}$.  And this is the characteristic screening length
for unbalanced charges.

\begin{definition}[Debye Length]
  The Debye length in a plasma is the characteristic screening length
  for an unbalanced charge, which is dictated by the kinetic energy of
  the plasma.
  \[ \lambda_{\rm D} = \qty[ \frac{\epsilon_0 \kbolt T}{ne^2}]^{\frac{1}{2}} \]
\end{definition}

\begin{table*}
    \begin{tabular}{l|cccc}
      Plasma         & Density ($\meter^{-3}$) & Electron temperature ($\kelvin$) & Magnetic Field ($\tesla$) & Debye Length ($\meter$) \\\hline
      Solar core     & $10^{32}$               & $10^7$                           & -                         & $10^{-11}$              \\
      Tokamak        & $10^{20}$               & $10^7$                           & 10                        & $2.10 \e{-5}$           \\
Hot interstellar gas & $10^6$                  & $10^4$                           & $10^{-10}$                & $10$ \\ \hline
    \end{tabular}
\caption{Properties of various plasmas.}
\label{tab:prop-plas}
\end{table*}
  
\begin{definition}[Plasma Parameter]
  The plasma parameter, $N$, is the number of particles of a plasma
  which are contained within the Debye sphere,
  \[ N = n \lambda_{\rm D}^3 \]
\end{definition}
For a good plasma we want $N \gg 1$. For the tokomak, $N \approx
1\e{6}$, for the interstellar gas, $N \approx 1\e{9}$. So, the
Interstellar medium's plasma is better than the tokomak's.

\subsection{Plasma Oscillation}
\label{sec:plasmaosc}

We know that the plasma is electrically neutral over scales around the
Debye Length, so there must be a restoring force driving the
restoration of charge neutrality. This will produce oscillations about
an equilibrium point (think of a swing).

In the simplest case is a perturbation in the electron number density,
holding the ions stationary, and ignoring thermal effects.  Here ions
represent a perfectly balancing positive charge density, matching the
electrons in equilibrium. \\
Let the electron number density be $n_{\rm e}(x,t)$, and suppose
$n_{\rm e}(x,t) = n_0 + \delta n(x,t)$; a perturbation consisting of a
constant, $n_0$, and a small fluctuating component, $\delta
n(x,t)$. As in any continuum, the evolution of the mass density of a
plasma is linked to the velocity field, and is given by the density
conservation law:
\begin{equation}
    \label{eq:density-conservation}
    \frac{\partial n}{\partial t} + \nabla \cdot (n_{\rm e} \vec u) = 0
\end{equation}
so the change in the electron density over time, plus the flux of
particles through a volume should be zero.

\begin{tikzpicture}[scale=0.5]
\draw [ultra thick, muted-green] (0,0) circle (3);
%\draw [accent-red, ultra thick, ->](-4,-4) -- (-2,-2);
\draw [accent-red, ultra thick, ->](4,1.7) node [black, right] {Flux in \dots} -- (2,1.7);
\draw [accent-blue, ultra thick, ->](2.5,0) -- (4,0) node[black, below, right, text width=3.5cm] {and Flux out are the only ways to change the internal density.};
\end{tikzpicture}

Now, a change in electron population or density, relative to the
equilibrium, produced by an electric field is
\[ \nabla \cdot \vec{E} = \frac{\rho_{\rm f}}{\epsilon_0} = \frac{-n_{\rm
    e}e}{\epsilon_0} \] with $\rho_{\rm f}$ the free-charge density.
How do electrons respond to the electric field? We need the fluid
momentum equation, which is a restatement of the conservation of
momentum,
\begin{equation}
  \label{eq:fluidmom}
  m \underbrace{\left[ \frac{\partial \vec{v}_{\rm e}}{\partial t} + ( \vec{v_{\rm e}} \cdot \nabla) \vec{v}_{\rm e} \right]}_{\frac{{\rm D} \vec{v}}{{\rm D}t}} = -e \vec{E}
\end{equation}
with $\frac{{\rm D} \vec{v}}{{\rm D}t}$ being the advective derivative,
\[ \frac{\rm D}{{\rm D}t} := \frac{\partial}{\partial t} + \vec{v}
\cdot \nabla \] Let us assume that the oscillation is a small
perturbation on an otherwise stationary (and therefore electric-field
free) equilibrium.
\[ n_{\rm e} = n_0 + \underbrace{n_1(x,t)}_{{\rm small}} \]
thus
\[ \vec{v} = \vec{v_0}+\vec{v_1}(\vec{x},t) \]
and taking $v_0 = 0$,
\[ \vec{E} = \vec{E_0} + \vec{E_1}(\vec{x}, t) \]
with $E_0 = 0$ since the field is in equilibrium.
We an now perturb the full equations to see how our small distribution evolves:
\[ \frac{\partial n_1}{\partial t} + \nabla \cdot (n_0 \vec{v_0}) =
0  \tag{\star}\]
So the momentum equation,
\[ \frac{\partial \vec{v_1}}{\partial t} = - \frac{e}{m} \vec{E_1}  \tag{\star \star \star}\]
and
\[ \nabla \cdot \vec{E_1} = - \frac{n_1 e}{\epsilon_0} \tag{\star \star}\]
Now, start by   $\frac{\partial \star}{\partial t} $,
\begin{align}
  \frac{\partial^2 n_1}{\partial t^2} + \nabla \cdot \left( n_0 
\frac{\partial \vec{v_1} }{\partial t} \right) &= 0 \\
\frac{\partial^2 n_1}{\partial t^2}+ \nabla \cdot \left( n_{0} \left( - \frac{e}{m} \vec{E_1} \right)\right) &=0 \\
\frac{\partial^2 n_1}{\partial t^2} + n_0 (- \frac{e}{m}) \nabla \vec{E_1} &= 0  \\
\frac{\partial^2 n_1}{\partial t^2} - \frac{n_0 e}{m} (- \frac{n_1 e}{\epsilon_0}) &= 0 \quad(\text{ by } \star \star)\\
\frac{\partial^2 n_1}{\partial t^2} + \frac{n_0e^2}{\epsilon_0 m} n_1 &= 0
\end{align}
\[ \ddot{n_1} + \omega_{\rm p}^2 n_1 = 0 \]
\[ \omega_{\rm p}^2 = \frac{n_0 e^2}{\epsilon_0 m} \]
Which is simple harmonic motion with fixed frequency $\omega_{\rm p}$, the plasma frequency, with $\nu_{\rm p} = 9 \sqrt{n_0}$. This is an oscillation, but not a wave.

\subsection{Plasma as a dielectric}
\label{sec:dielectric}

The plasma oscillation has consequences for the propogation of
electromagnetic radiation. The restoring force which produces the
plasma is a direct consequence of the plasma producing a displacement
current. It turns out that we can treat the plasma as a dielectric
medium, and that we can see this by considering the plasma's repsonse
to an oscillating imposed electric field,
\[ E(t) = \hat{E} e^{-i\omega t} \]
the plasma responce: consider a single particle,
\begin{equation} m \frac{\dif{v}}{\dif{t}} = -eE = -e \hat{E} \exp({-i \omega t})
\end{equation}
therefore, 
\[ v(t) = \frac{e}{i \omega m} \hat{E} \exp(-i \omega t) = \frac{e}{i
  \omega m} E(t)\] and the plasma particles oscillate in response, but
not at the same phase. Charges in motion constitute a current, so for
the current density we can write that
\begin{align} \vec{j} &= -n e \vec{v} = -ne \left( \frac{e}{i \omega m} \right) \vec{E}(t) \nonumber
 \\ &= \frac{ne^2}{i \omega m}\vec{E}(t) 
\end{align}
Recall Maxwell's equations for a dielectric:
\begin{align*}
  \nabla \times \vec{H} = \frac{\partial \vec D}{\partial t}
\end{align*}
with $\vec{D}= \epsilon_{{\rm r}} \epsilon_0 \vec{E}$, and $\epsilon_{\rm r}$ the relative permittivity of the dielectric.
The full Maxwell equation reads
\begin{align*}
  \nabla \times \vec{H} &= \vec{j} + \frac{\partial \vec D}{\partial t}
\end{align*}
in the plasma, and since $\vec{D} = \epsilon_0 \vec{E}$,
\begin{align*}
  \nabla \times \vec{H} = - \frac{n e^2}{i \omega m} \vec{E} - i \omega \epsilon_0 \vec{E}
\end{align*}
and so $\vec{E} \propto \exp(-i \omega t)$, and so,
\begin{align}
  \nabla \times \vec{H} &= - i \omega \left[ 1 - \frac{n e^2}{\epsilon_0 m \omega^2} \right] \epsilon_0 \vec{E} \\ &= -i \omega \epsilon \epsilon_0 \vec{E} \nonumber
\end{align}
just like a dielectric, where 
\begin{equation}
  \label{eq:epsilon}
  \epsilon = 1 - \frac{\omega^2_{\rm p}}{\omega^2}
\end{equation}
is the plasma dielectric constant.
NB $\epsilon = \epsilon(\omega)$; what's the connection with refractive index?
\begin{align*}
  \nabla \times \vec{H} &= -i \omega \epsilon \epsilon_0 \vec{E} \\
\nabla \times (\nabla \times \vec{H}) &= -i \omega \epsilon \epsilon_0 \nabla \times \vec{E} \\ &= \nabla(\nabla \cdot H) - \nabla^2 H \\ &= - \nabla^2 \vec{H}
\end{align*}
since $\nabla \cdot \vec{H} = 0$ in a plasma.
So, using a complex notation for waves. Now, $\nabla \times \vec{E} = - \frac{\partial \vec{B}}{\partial t}$, and we will take $\vec{B}= \mu_0 \vec{H}$, so,
\begin{align*}
  - \nabla^2 \vec{H} &= -i \omega \epsilon \epsilon_0 \nabla \times \vec{E} \\ &= i \omega \epsilon \epsilon_0 \mu_0 \frac{\partial \vec{H}}{\partial t}
\end{align*}
or 
\begin{equation}
  \label{eq:waveeqpropeminplasma}
  \nabla^2 \vec{H} + \frac{\omega^2}{c^2} \epsilon \vec{H} = 0
\end{equation}
which is the wave equation  for the propagation of electromagnetic waves in the plasma.
There is then a dispersion relation,
\begin{align*}
  \frac{\partial}{\partial t} & \to -i \omega \\
\vec{\nabla} & \to i \vec{k}
\end{align*}
so
\begin{align*}
  -k^2 + \frac{\omega^2}{c^2}\epsilon &= 0 \\
\frac{\omega}{k} &= \frac{c}{\epsilon^{\frac{1}{2}}}
\end{align*}
so, $\epsilon^{\frac{1}{2}}$, and the plasma refractive index is then
\begin{equation}
  \label{eq:plasmarefind}
  n_{\rm plasma} = \left[ 1 - \frac{\omega_{\rm p}^2}{\omega^2}\right]^{\frac{1}{2}}
\end{equation}
Hence, if $\omega < \omega_{\rm p}$, the index is purely imaginary,
and there is no wave propagation. If $\omega > \omega_{\rm p}$ waves
can propagate, but they will be affected. If $\omega = \omega_{\rm
  p}$---as we see more closely in the full cold plasma treatment, this
represents wave absorption. The full dispersion relation can then be
written
\begin{equation}
  \label{eq:plasmadispersion}
  \omega^2 = \omega_{\rm p}^2 + k^2c^2
\end{equation}

\section{Cold Magnetised Plasma Model}
\label{sec:coldmagplasm}
We will generalise the dielectric concept into a plasma immersed in a
uniform magnetic field, but again, we will ignore thermal fluctuations
in comparison with other dynamics, i.e.\ a cold plasma. A cold plasma
doesn't need to have a low temperature, but must have the
termodynamic-based dynamics dominated by another factor. \\
Recall what happens when a moving charged particle encounters a uniform magnetic field, 
\begin{equation}\label{eq:momentumelectron}
  m \vec{\dot{v}} &= q(\dot{\vec{v}} \times \vec{B})
\end{equation}
and assume that $\vec{B}$ lies in the $\vec{\hat{z}}$-direction, then
\[ \vec{B} = \vec{\hat{z}} B_0 \]
so \[ \dot{\vec{v}} = \frac{q}{m} \left[ \vec{v} \times ( \hat{\vec{z}} B_0) \right] \]
so
in components,
\[ \dot{v_x} = \frac{q}{m} \left[ \dot{v_y}B_0 - \dot{v}_z 0\right] = \frac{q B_0}{m} v_y\] 
\[ \dot{v_y} = \frac{q}{m} \left[ \dot{v_z} 0 - \dot{v}_x B_0\right] = - \frac{q B_0}{m} v_x\] 
\[ \dot{v_z} = \frac{q}{m} 0 = {\rm const} \]
so, defining the cyclotron frequency
\begin{equation}
  \label{eq:cyclotron}
  \omega_{\rm c} = \frac{|q|B}{m}
\end{equation}
and for electrons, $\nu_{\rm c} = 28\ \giga \hertz\ \tesla^{-1}$,
hence
\[ \ddot{v}_{x} = \omega_{\rm c} \dot{v_y} = \omega_{\rm c} [-\omega_{\rm c} v_x] \]
that is
\[ \ddot{v}_x + \omega_{\rm c}^2 v_x = 0 \] which is plane
perpendicular to the charged particle (e.g. an electron), which will
thus undergo circular motion. THis motion is, however, uniform along
the field. The net effect is that the particle will describe a helix.\\
{\em \textbf{Exercise}: Show that the magnetic field doesn't change the particle's energy. Hint, consider $\vec{v} \cdot \dot{\vec{v}}$}\\
Let's now consider the general response of a plasma in both magnetic
and electric fields. We need to include the positive ions and the
electrons together, plus, since this is a plasma, we need to consider
the collective effects, i.e. a fluid treatment.
Let's define the perturbation,
\begin{align*}
n_{\rm s} & = n_0_{\rm s}+ n_1(\vec{x},t) \\
\vec{v_{\rm s}} & = \vec{v_{0,{\rm s}}} + \vec{v_1}(\vec{x},t)
\end{align*}
Now, under perturbation, we can linearlise the most important equations:
\begin{subequations}
\begin{align}
  \frac{\partial n_{\rm s} }{\partial t} + \nabla \cdot \qty(n_{\rm s} \vec{v_{\rm s}}) &= 0 \\
  \dot{n_{\rm s}} + n_{0,{\rm s}} \nabla \cdot \vec{v}_{\rm s} &= 0 
\end{align}
\end{subequations}

\begin{subequations}
\begin{align}
\frac{\partial v_{\rm s}}{\partial t} +  (\vec{v_{\rm s}} \cdot \nabla) \vec{v_{\rm s}} &= \frac{q_{\rm s}}{m_{\rm s}} \qty[ \vec{E_{\rm s}}+ \vec{v_{\rm s}}\times \vec{B} ]  \\  
\dot{\vec{v}} &= \frac{q_{\rm s}}{m_{\rm s}} [ \vec{E_1}+\vec{v_{\rm F}}\times B_0 ] 
\end{align}
\end{subequations}
\begin{subequations}
\begin{align}
\vec{J} &= \sum_{\rm s} n_{\rm s} q_{\rm s}\vec{v_{\rm s}} \\ 
J_1 &= \sum_{\rm s} n_{0,{\rm s}} q_s \vec{v_{1, \rm s}}
\end{align}
\end{subequations}
Recall Maxwell's equations
\begin{align}
  \nabla \times \vec{B} &= \mu_0 \vec{J} + \frac{1}{c^2} \frac{\partial \vec{E}}{\partial t} \\
\nabla \times \vec{E} &= - \frac{\partial \vec{B}}{\partial t} \\
\nabla \cdot \vec{E} &= \frac{\rho_{\rm f}}{\epsilon_0} \\
\nabla \cdot \vec{B} &= 0
\end{align}
{\em Tactic}: If we have $\vec{v}$ in terms of $\vec{E}$, ($\vec{v}(E)$), then, $\vec{J} = \vec{J}(\vec{v}) = \vec{J}(\vec{E})$, with $\vec{J} = \vec{\sigma} \vec{E}$, then if we have a conductivity law we can move to a dielectric description.

\begin{align*}
  \vec{v} &= \frac{q}{m} \qty[ \vec{E} + \vec{v} \times \vec{B_0}]\\
-i \omega v_x &= \frac{q}{m} \qty[ E_x + v_yB_0] \\
&= \frac{q}{m}E_x + \omega_cv_y \\
-i \omega v_y &= \frac{q}{m}E_y - \omega_cv_x
\end{align*}
so
\begin{align*}
  -i\omega v_x &= \frac{q}{m} E_x - \frac{\omega_c}{i \omega} \qty[ \frac{q}{m}E_y - \omega_cv_x] \\
-i \omega v_x - \frac{\omega_c^2}{i \omega} v_x &= \frac{q}{m} E_x - \frac{q \omega_c}{i \omega m}E_y \\
&= \frac{q}{m}\qty[E_x - \frac{\omega_c}{i \omega} E_y] \\
\qty(1 - \frac{\omega_c^2}{\omega^2}) v_x &= - \frac{q}{i \omega m} \qty[E_x - \frac{\omega_c}{i \omega}E_y] \\
\text{since } \qty(1- \frac{\omega_c^2}{c^2}) v_y &= - \frac{q}{i \omega m} \qty[ E_x + \frac{\omega_c}{i \omega} E_y] \\
(1 - \frac{\omega_c^2}{\omega^2} \vec{v} &= M \cdot \vec{E} \\
\vec{v} &= \frac{1}{\qty(1 - \frac{\omega_{c}^2}{\omega^2})} \cdot M \cdot \vec{E}
\end{align*}
So, we know that we can get to this expression for $\vec{J}$---how does this help? We now bring Maxwell's equations into the mixture.
\begin{align*}
 \nabla \times \vec{E} &= \pdv{\vec{B}}{t} \\
\nabla \times ( \nabla \times \vec{E}) &= - \nabla \times \qty( \pdv{\vec{B}}{t}) \\
&= - \pdv{t} \qty(\nabla \times \vec{B}) \\
&= - \pdv{t} \qty[\mu_0 \vec{J} + \frac{1}{c^2} \pdv{\vec{E}{t}}] \\
&= - \pdv{t} \qty[\mu_0 \vec{\sigma} \cdot \vec{E} + \frac{1}{c^2} \pdv{\vec{E}}{t}]
\end{align*}
We are interested in waves, where solutions are proportional to $e^{i \vec{k}\cdot \vec{r} - i \omega t}$ \\
LHS: \\
\begin{align*}
  \nabla \times (\nabla \times \vec{E}) &= - \vec{k} \times (\vec{k} \times \vec{E})
\end{align*}
RHS: \\
\begin{align*}
  i \omega \qty[ \mu_0 \vec{\sigma} \cdot \vec{E} - \frac{i \omega}{c^2} \vec{E}]
\end{align*}
So, the full equation is
\[ \vec{k} \times (\vec{k} \times \vec{E}) + \frac{\omega^2}{c^2} K \cdot \vec{E} &= 0 \]
with
\[ K = I + \frac{i \sigma}{\epsilon_0 \omega} \] being the dielectric
tensor.\\
Define the generalised refractive index,
\[ \vec{n}= \frac{\vec{k}c}{\omega} \]
(a refractive index with ``directional complications''),
\begin{equation}\label{eq:genrefind} \vec{n} \times ( \vec{n} \times \vec{E} ) + K \cdot \vec{E} = 0
\end{equation}
To help to understand the significance of equation (\ref{eq:genrefind}), let's choose a geometry---let's put $\vec{B}_0 = \vec{\hat{z}} B_0$, and let's take a wave in the $\vec{\hat{x}}-\vec{\hat{z}}$-plane (with one component parallel to $\vec{B}_0$, and one perpendicular),
\begin{align*}
  \vec{n} &= \hat{x} n \sin(\theta) + \hat{z} n \cos(\theta)
\end{align*}
Then expand the vector cross-product $\vec{n} \times (\vec{n} \times \vec{E})$ for this choice of geometry.
\begin{align}
  \begin{pmatrix}
S-n^2 \cos^2\theta  & - iD & n^2 \cos(\theta) \sin(\theta) \\
iD & S-n^2 & 0 \\
n^2 \cos\theta \sin\theta & 0 & P- n^2 \sin^2 \theta
\end{pmatrix}
\begin{pmatrix}
  E_x \\ E_y \\ E_z
\end{pmatrix}
= 0
\end{align}
Where we have written
\begin{equation}
  \label{eq:dielectricten}
K=
  \begin{pmatrix}
  S & -iD & 0 \\ iD & S & 0 \\ 0 & 0 & P  
  \end{pmatrix}
\end{equation}
Where
\begin{align*}
  S &= \frac{1}{2}(R+L) \\
  D &= \frac{1}{2}(R-L)
\end{align*}
and 
\begin{align*}
  R &= 1 - \sum_S \frac{\omega^2_{\rm P_s}}{\omega^2} \qty( \frac{\omega}{\omega+\epsilon_0 \omega_{\rm c_s}}) \\
&= 1 - \frac{\omega_{\rm P}^2}{(\omega+\omega_{\rm C_+})(\omega-\omega_{\rm c^-})} \\
L &= 1 - \sum_S \frac{\omega_{\rm P_s}^2}{\omega^2} \qty(\frac{\omega}{\omega-\epsilon_{\rm s} \omega_{\rm c_s}}) \\
&= 1 - \frac{\omega_{\rm p}^2}{(\omega-\omega_{\rm c_+})(\omega+\omega_{\rm c_-})} \\
\epsilon_{\rm s} &=
\begin{cases}
  +1 & \text{ for positive ion } \\
  -1 & \text{ for negative electron}
\end{cases} \\
\omega_{\rm P}^2 &= \omega_{\rm P_+}^2 + \omega_{\rm P_{-}}^2 \\
P &= 1 - \frac{\omega_{\rm P}}{\omega^2}
\end{align*}
For a non-trivial electric field the determinant of the matrix must vanish, giving a relationship between $\omega$ and $k$ so th dispertion relation 
\begin{equation}
  \label{eq:dispertionrelation}
  An^4 - Bn^2 + C = 0
\end{equation}
with
\begin{align*}
  A &= S \sin[2](\theta) + P \cos[2](\theta) \\
B &= RL \sin[2](\theta) + PS (1 + \cos[2](\theta) \\
C &= PRL
\end{align*}
{\em N.B. There are two spherical cases, $\theta=0$, and $\theta= \frac{\pi}{2}$}.\\
For the case $\theta=0$, the propagation is parallel to $B_0$, so
\begin{equation*}
  \begin{pmatrix}
S-n^2 & -iD & 0 \\
iD & S-n^2 & 0 \\
0 & 0 & P 
\end{pmatrix}
\begin{pmatrix}
  E_x \\ E_y \\ E_z
\end{pmatrix} = 0
\end{equation*}
i.e. $(S-n^2)^2 - D^2 = 0$ if $E_x, E_y \neq 0$ or $ P=0$ if $E_z\neq
0$.\\
When $P=0$ we have longitudinal plasma oscillations (just like before),
\begin{align*}
  n^2 = R & \frac{iE_x}{E_y} = - \frac{S-n^2}{D} = - \frac{S-R}{D} = 1 & \text{right circ. pol.} \\
n^2 = L & \frac{i E_x}{E_y} = - \frac{S-n^2}{D} = - \frac{S-L}{D} = -1 & \text{left circ. pol.}
\end{align*}
{\em N.B. Recall that $S = \frac{1}{2} (R+L) \therefore S-R = \half{}(-R+L) = -D$ etc.} \\
When $n^2 = R, L$ we get circularly polarised transverse waves.\\
$n^2=R$ has resonance at $\omega = \omega_{\rm c_e}$, $n^2=L$ at $\omega=\omega_{\rm c_i}$, which are the particle cyclotron frequencies where the particle naturally oscillates.
$R, L$ has a cut-off when the numerator is zero,
\begin{equation}
  \label{eq:cutoff}
  \omega^2 \mp (\omega_{\rm re}-\omega_{\rm c_i}) \omega - (\omega_p^2 + \omega_{c_i}\omega_{c_e})=0
\end{equation}
i.e. at $\omega \sim \mp \half \omega_{\rm c_e}+ \qty(\omega_{\rm p}^2 + \frac{1}{4}\omega_{c_{e}}^2)^{\frac{1}{2}}$.
At the low-frequency limit, if $\omega \ll \omega_{\rm c_i}$ (the lowest natural frequency is $\omega_{\rm c_i}$, then $R \sim L \sim 1+ \frac{c^2}{c_A^2}$, where $C_A^2 = \frac{B_0^2}{\mu_0\rho_0}$ is the Alfven speed. The refractive index is $n^2 = 1+\frac{c^2}{c_A^2}$, i.e. $\omega^2 = \frac{k^2 c^2}{1 + \frac{c^2}{c_A^2}} \approx k^2c_A^2$ which descripe non-dispersive Alfven waves (c.f. Magnetohydrodynamics later).

For $\theta = \frac{\pi}{2}$, waves propagating perpendicular to $B_0$, the dispertion relation becomes 
\[ Sn^4 - (RL + PS)n^2 +PRL = 0 \]
with two roots at $n^2=P$, and $n^2 = \frac{RL}{S}$.
Characteristics of these modes are best seen in a matrix system.
\begin{equation}
\label{eq:dispertion}
  \begin{pmatrix}
    S  & -iD   & 0 \\
    iD & S-n^2 & 0 \\
0      & 0     & P-n^2
  \end{pmatrix}
  \begin{pmatrix}
    E_x \\ E_y \\ E_z
  \end{pmatrix} = 0
\end{equation}
\begin{enumerate}
\item For $E_x = E_y = 0$, and $E_z \neq 0$, we have $n^2=P$, i.e.  we have the same dispertion relation as in the unmagnetic case. Note, $\vec{k}\cdot \vec{E} = 0$, so we have transverse electormagnetic waves which are independent of $B_0$, and $E$ is parallel to $B_0$, so all motion is aligned with $B_0$, and therefore
\[ n^2 = P \qquad \text{(Ordinary mode, O-mode)}\]
and the cutoff is at the plasma frequency, with no resonance.
\item Now suppose we consider $E_z=0$, $E_x, E_y \neq 0$, then $n^2 = \frac{RL}{S}$ in solution,
  \begin{equation*}
    \frac{iE_x}{E_y} = - \frac{S-n^2}{D} = - \frac{D}{S}
  \end{equation*}
(first row gives $SE_x - iD E_y = 0$). Hence the wave is partly longitudinal, and partly transverse, since both $E_x, E_y \neq 0$, and thus $\vec{k} \cdot \vec{E} \neq 0$. The fact that $E$ is perpendicular to $B_0$ means that same for the gyration about the magnetic field which is generated, so wave properties depend on $B_0$,
\[ n^2 = \frac{RL}{S} \qquad \text{(Extraordinary mode, X-mode)}\]
the cutoff $R=0$ or $L=0$, and resonance at $S=0$. 
\end{enumerate}
We have
\begin{align*}
  R & = 1 - \frac{\omega_{\rm p}^2}{(\omega+\omega_{\rm c_+})(\omega-\omega_{c_-})} \\
  L & = 1 - \frac{\omega_{\rm p}}{(\omega-\omega_{\rm c_+})(\omega+\omega_{\rm c_-})}
\end{align*}
since $\omega \ll \omega_{c_i}$
\begin{align*}
  R &\approx 1 - \frac{\omega_p^2}{\omega_{c_i}(-\omega_{c_e})} \\
    &= 1 + \frac{\frac{ne^2}{\epsilon_0 m_e} + \frac{n e^2}{\epsilon_0 m_i} } {\frac{eB_0}{m_i} \frac{e B_0}{m_e}} \\
    &= 1 + \frac{n(m_i+m_e)}{\epsilon_0 B_0^2} \\
    &\approx 1 + \frac{\mu_0 \rho_0 c^2}{B_0^2} \\
    &= 1 + \frac{c^2}{c_A^2} \approx \frac{c^2}{c_A^2}
\end{align*}
For $\theta = 0$, $\omega \ll \omega_{c_i}$, which are both circular polarisations for transverse Alfven waves. Then, 
\[ n^2 = \frac{kL}{S} \therefore \omega^2 = k^2 c_A^2 \]
and 
\[ S = \half (R+L) \]
but $n^2 = P$, so there is a cutoff; no low frequency is possible.\\
For $n^2 = \frac{RL}{S}$ (the X-mode) at low frequency.  We have a
compressional Alfven wave, which is different to the $\theta=0$ case,
even though they share a dispersion relation.\\
For $\theta=\frac{\pi}{2}$,
\begin{equation*}
  \begin{pmatrix}
    S & -iD & 0 \\
iD & S-n^2 & 0 \\
0 & 0 & P-n^2
  \end{pmatrix}
\end{equation*}
So
\begin{align*}
  S E_x - iD E_y & = 0 \\
 \frac{i E_x}{E_y} &= - \frac{D}{S}
\end{align*}
So what are the implications for the Alfven wave? Since $R$, $L$ are approximately equal, for $\omega \ll \omega_c$,
\[ \qty|\frac{E_x}{E_y}| \ll 1 \]
and so $|E_y| \gg |E_x|$. From the equations of motion, 
\begin{align*} 
\dot{v}_x &= - \frac{e}{m} E_x - \omega_c v_y \\
\dot{v}_y &= - \frac{e}{m} E_y - \omega_c v_x
\end{align*}
we can differentiate the system with respect to $t$ to show
\begin{equation*}
  \qty| \frac{v_x}{v_y} | \approx \frac{\omega_c}{\omega} \quad \text{when}\quad \qty| \frac{E_x}{E_y} | \approx 0
\end{equation*}
We know that the O-mode cuts off at $\omega=\omega_{\rm p}$. For the
X-mode, $n^2= \frac{RL}{S}$, the two cutoffs occur at $R=0$ or
$L=0$. This is the same as the circuarly polarised waves as before,
but we have two cutoff frequencies for the same mode.  Resonance
occurs at $S=0$, and again, this occurs at two places, an upper and a
lower hybrid frequency:
\begin{align*}
  \omega_u^2 &= \omega_p^2 + \omega_{c_e}^2 \\
\omega_l^2 &= \omega_p^2 \frac{\omega_{c_i}\omega_{c_e}}{\omega_p^2+\omega_{c_e}^2}
\end{align*}

\section{Ideal Magnetohydrodynamic Plasmas}
\label{sec:idealmhd}

Here we consider the behaviour of plasmas at long wavelengths, and low
frequencies; in this limit we can retrieve classical thermodynamic
relations. 
Starting with the model equations,
\begin{subequations}
  \begin{align}
    \pdv{\rho}{t} + \nabla \cdot (\rho \vec{u}) &= 0 \\
\rho \adv{\vec{u}}{t} &= - \nabla p + \vec{J} \times \vec{B} + (q \vec{E}) \\
\adv{t} \qty[ p \rho^{-\gamma} ] &= \frac{2}{3} \rho^{-\gamma} \qty[ \vec{J} - q \vec{u}] \cdot \qty[ \vec{E} + \vec{u} \times \vec{B}] \\
\vec{J} &= \sigma \qty[ \vec{E} + \vec{u} \times \vec{B}]
  \end{align}
\end{subequations}
with $\rho$ the mass density, $\vec{u}$ the bulk fluid velocity field,
$p$ the scalar pressure, $q$ the unbalanced charge, $\vec{J}$ the
current density, and $\vec{B}$
the magnetic field. We take $\gamma = \flatfrac{5}{3}$\\
We assume there is no $\vec{E}$ in the momentum equation, partly
because we assume the plasma is a single species fluid with all the
electrical properties of an electron-ion plasma, but with no charge
separation. Additionally, we can consider the electric field as
arrising only due to frame changes. \\
In this situation, Maxwell's equations are
\begin{subequations}
  \begin{align}
    \nabla \times \vec{B} &= \mu_0 \vec{J} \\
\nabla \times \vec{E} &= - \pdv{\vec{B}}{t} \\
\nabla \cdot \vec{B} &= \nabla \cdot \vec{E} =0
  \end{align}
\end{subequations}
Notably, Ampere's law contains no mention of displacement current, and
there is no allowance for charge separation.
An ideal MHD plasma exhibits perfect conductivity, so
\[ \vec{E} + \vec{u} \times \vec{B} = \eval{\frac{\vec{J}}{\sigma}}_{\sigma \to \infty} \] 
thus
\[ \vec{E} + \vec{u} \times \vec{B} =0\]
and
\[ \adv{t}\qty[p \rho^{-\gamma}] = 0\] In order to understand how the
ideal MHD plasma behaves we need to look at the normal modes, just as
with the cold plasma case.
Take a perturbation,
\begin{subequations}
  \begin{align}
    p   & = p_0 + p_1 (\vec{r}, t)             \\
\vec{B} & = \vec{B}_0 + \vec{B}_1 (\vec{r}, t) \\
\vec{u} & = \vec{u}_0 + \vec{u}_1 (\vec{r}, t) \\
\rho    & = \rho_0 + \rho_1 (\vec{r}, t)       \\
\vec{J} & = \vec{J}_0 + \vec{J}_1 (\vec{r}, t)
  \end{align}
\end{subequations}
Then we assume $\vec{u}_0 = 0$ (stationary equilibrium), and
$\vec{J}_0 = \frac{\nabla \times \vec{B}}{\mu_0} = 0$.

Then linearise,
\[ \pdv{\rho}{t} + \nabla \cdot (\rho \vec{u}) = 0 \to \pdv{\rho_1}{t}
+ \vec{u}_0 \cdot \rho_1 = 0 \] then, assuming all perturbed
quantities are proportional to $\exp[ i(\vec{k} \cdot \vec{r} - \omega
t)]$, so
\begin{align*}
  \rho_1 &= \frac{\rho_0}{\omega} (\vec{k} \cdot \vec{u}_1) \tag{I} \\
\rho_0 &= \pdv{\vec{u}_1}{t} = - \nabla p_1 + \vec{J}_1 \times \vec{B}_0 
\end{align*}
Wave analysis on the momentum equation leads to
\begin{align*}
  \omega \rho_0 \vec{u}_1 &= \vec{k} p_1 + \frac{\vec{B}_0 \cdot \vec{B}_1}{\mu_0} \vec{k} - \frac{(\vec{k} \cdot \vec{B}_0)}{\mu_0} \vec{B}_1 \tag{II} \\
\pdv{\vec{B}}{t} &= - \nabla \times \vec{E} = \nabla \times (\vec{u}_1 \times \vec{B}_0) \\
\omega B_1 &= ( \vec{k} \cdot \vec{u}_1) \vec{B}_0 - (\vec{k} \cdot \vec{B}_0 ) \vec{u}_1 \tag{III} \\
\end{align*}
Since $p \rho^{-\gamma}$ is constant,
\[ \adv{t} \qty( p \rho^{-\gamma} ) \]
so
\begin{equation*}
  p_1 = c_{\rm s}^2\rho_1 \tag{IV}
\end{equation*}
where $c_{\rm s}^2$ is the sound speed in the plasma, and since $\nabla \cdot \vec{B} = 0$,
\begin{equation*}
  \vec{k} \cdot \vec{B}_1 = 0 \tag{V}
\end{equation*}
We now want to eliminate $p_1$ and $B_1$ from (II), to obtain
everything in terms of $u_1$. $p_1$ can be eliminated using (IV),
$\rho_1$ using (I), and $B_1$ using (III).

This process yields
\begin{equation}
  \label{eq:mhdequation}
\begin{split}
  \qty[ \omega^2 - \frac{(\vec{k} \cdot \vec{B}_0 )^2}{\mu_0 \rho_0}] \vec{u}_1 = \qty[(r_{\rm s}^2 + c_{\rm A}^2) \vec{k} - \qty( \frac{\vec{k} \cdot \vec{B}_0}{\mu_0 \rho_0} ) B_0] (\vec{k} \cdot \vec{u}) \\- \frac{(\vec{k}\cdot \vec{B}_0)(\vec{B}_0 \cdot \vec{u}_1)}{\mu_0 \rho_0} \vec{k}
\end{split}
\end{equation}
with $c_{\rm A}^2 = \frac{B_0^2}{\mu_0 \rho_0}$.
Now we choose 
\begin{subequations}
  \begin{align}
    \vec{B}_0 &= B_0 \hat{b} \\ \vec{k} &= k \hat{z}
  \end{align}
\end{subequations}
Then,
\begin{equation}
  \label{eq:mhdincoordis}
\begin{split}
  \qty[ \omega^2 - k^2 c_{\rm A}^2 (\hat{z} \cdot \hat{b})^2 ] \vec{u}_1 = \qty[ k^2 (c_{\rm s}^2 + c_{\rm A}^2) \hat{z} + \vec{k}^2 c_{\rm A}^2 (\hat{z} \cdot \hat{b} ) \hat{b} ](\hat{z} \vec{u}_1) \\ - k^2 c_{\rm A}^2 (\hat{z} \cdot \hat{b}) (\hat{b} \cdot \vec{u}_1) \hat{z}
\end{split}
\end{equation}
Let's consider a special case; In MHD $\omega$ is much less than the
minimum cyclotron frequency for ions and plasma frequency, and the
wavelength is much greater than the Debye length of the Larmour
Radius. Consider the component of $\vec{u}_1$ perpendicular to the
direction of motion, $\hat{z}$. To find this direction take the cross-product of the whole equation (\ref{eq:mhdincoordis}), with $\hat{z}$. Now take $\hat{z} \cdot \hat{b} = \cos(\theta)$.
\begin{align*}
  (\omega^2 - k^2 c_A^2 \cos[2](\theta))(\hat{z} \times \vec{u}_1) &= -k^2 c_A^2 \cos(\theta) (\hat{z} \cdot \vec{u}_1)(\hat{z} \times \hat{b})
\end{align*}
Suppose the plasma is incompressible, so $\hat{z} \cdot \vec{u}_1 = 0$, and we still have $(\omega^2 - k^2 \cos[2](\theta))(\hat{z} \times \vec{u}_1) = 0$. Clearly a non-trivial solution, governed by the dispertion relation is possible: a wave which satisfies
\begin{equation}
  \label{eq:shearalfven}
  \omega^2 = k^2 c_A^2 \cos[2](\theta) 
\end{equation}
which is an \emph{shear Alfv\'{e}n wave}.  We have a transverse wave
at low frequency--for $\theta=0$ this is just like a cold plasma
solution, but at $\theta=\frac{\pi}{2}$ there is no transverse wave
solution, but the cold plasma has the low frequency limit of the X-mode, the compressed Alfv\'en mode. 

The general solution for MHD waves is that we have a dispertion
reltion, 
\[ \qty(\omega^2 - k^2 c_A^2 \cos[2](\theta) ) \cdot \qty( \omega^4 - k^2( c_A^2 + c_s^2) \omega^2 + k^2c_s^2c_A^2 \cos[2](\theta)) \]
Then there are three modes,
\begin{itemize}
\item Alfven \[ \omega^2 = k^2 c_A^2 \cos[2](\theta) \]
\item Fast (+) and Slow (-) Magnetosonic \[ 2 \frac{\omega^2}{k^2} = c_s^2 + c_A^2 \pm \sqrt{( c_s^2 + C_A^2)^2 - 4k^2 c_s^2 c_A^2 \cos[2](\theta)} \]
\end{itemize}
For the fast MS mode, $\frac{B^2}{2 \mu_0}$ magnetic pressure enhanes
the thermodynamic pressure, $p$, by varying in phase with it.  Fr the
slow MS mode the magnetic presure and thermodynamic pressure oppose
one another. Magnetic pressure plays a powerful conceptual role in
MHD.

Recall the equilibrium; to see how significant the magnetic pressure can be,
\begin{align*}
  \adv{t} &= 0\\
\nabla p_0 &= \vec{J}_0 \times \vec{B}_0 
\end{align*}
before we set $\vec{J}_0 =0$ (so the plasma had a uniform $\vec{B}_0$ and $p_0$, however, we can generalise this, since 
\begin{align*}
  \nabla \times \vec{B} &= \mu_0 \vec{J}_0 \\
\nabla p_0 &= \frac{1}{\mu_0} (\nabla \times \vec{B}_0) \times \vec{B}_0
\end{align*}
which can be rewritten in the form
\[ \nabla \qty(p_0 + \half \frac{\vec{B}_0^2}{\mu_0}) +
\frac{1}{\mu_0}( \vec{B}_0 \cdot \nabla) \vec{B}_0 \]
for the simplest geometry take $\vec{B}_0 \cdot \nabla \vec{B}_0 = 0$, for equilibrium, 
\begin{equation}
  \label{eq:equilibrium}
  \nabla \qty(p_0 + \frac{\vec{B}_0^2}{2 \mu_0} ) = 0
\end{equation}
That is, the plasma pressure plus the magnetic pressure is constant.
Plasma tends to avoid the strongest field regions in equilibrium. This is a possible mechanism for confinement. How could this go wrong? Suppose we have a cylinder of plasma, which is carrying current. It has an azimuthal $\vec{B}_0$

% \begin{verbatim}
%      /     /      /
%  ----------------|--------------
%      |     |     |      ----> J$_0$
%      |     |     |
%  ----\-----\-----\-------------
% \end{verbatim}

\begin{center}
  \begin{tikzpicture}[]

	\begin{scope}
		\fill [shading=axis, bottom color=accent-red!30, top color=accent-red!40, middle color=accent-red!20] (2,0) ellipse (.2 and .5);

		\fill [shading=axis, bottom color=accent-red!30, top color=accent-red!40, middle color=accent-red!20, ] (-2,-.5) rectangle (2, .5);
		\fill [fill=accent-red!20] (-2,0) ellipse (.2 and .51);
	\end{scope}

        \begin{scope}
          \foreach \x in {-1,-.5,..., 1.5}{ \draw [accent-blue, ->]
            (\x, -.5) ..controls (\x+.2, -.3) and (\x+.2, .3)
            .. (\x,.5); } 
\draw [accent-blue, ->]
            (-1.8, -.5) ..controls (-1.8+.2, -.3) and (-1.8+.2, .3)
            .. (-1.8,.5) node [midway, right] {$\vec{B}_0$};
        \end{scope}

	\draw [help lines, ->] (0,.9) -- (2,.9) node [midway, fill=white] {$\vec{J}$};

\end{tikzpicture}
\end{center}
If we bend the cylinder, but wish to maintain $\vec{J}_0$,
% \begin{verbatim}
%        weak B$_0$
%     ____________
%   -/    ________
%  /     /  strong B$_0$
% /     /
% \end{verbatim}
% \begin{center}
%   \begin{tikzpicture}[]

% 	\begin{scope}[scale=0.5]
% %		\fill [shading=axis, bottom color=accent-red!30, top color=accent-red!40, middle color=accent-red!20] (2,-1) ellipse (.2 and .5);

% \foreach \y in {2}{
% 	\pgfmathsetmacro{\yp}{\y-3}
% 		\fill [shading=axis, bottom color=accent-red!30, top color=accent-red!40, middle color=accent-red!20] 
%         (7,\yp) .. controls (3,\yp) and (2,\y) .. (-1,\y) node [pos=0.5] (B) {} node [pos=0.4] (C)--
%         (-1,\y-2) .. controls (2,\y-2) and (3,\yp-2) .. (7, \yp-2) node [pos=0.5] (A) {} node [pos=0.6] (D) --cycle;
% }
% \draw (A) -- (B) (C) -- (D);
% 		\fill [fill=accent-red!20] (-1,1) ellipse (.2 and 1);
% \fill [shading=axis, bottom color=accent-red!30, top color=accent-red!40, middle color=accent-red!20] (7,-2) ellipse (.2 and 1);
% 	\draw [help lines, <-] (7,-2) .. controls (3,-1) and (2,1) .. (-1,1) node [midway, below] {$\vec{J}$};
% 	\end{scope}

%          \begin{scope}[scale=0.5]
%            \foreach \x in {-1,-.5,..., 1.5}{ \draw [accent-blue, ->]
%              (\x, -.5) ..controls (\x+.2, -.3) and (\x+.2, .3)
%              .. (\x,.5); } 
%  \draw [accent-blue, ->]
%              (-1.8, -.5) ..controls (-1.8+.2, -.3) and (-1.8+.2, .3)
%              .. (-1.8,.5) node [midway, right] {$\vec{B}_0$};
%          \end{scope}


% \end{tikzpicture}
% \end{center}
The magnetic pressure is stronger on the inside of the bend than the
outside. Thus the plasma is push upwards, wosening the distortion. This gives a kink instability.

\subsection{Pinch Instability}
\label{sec:pinch}

% \begin{verbatim}

% --------------\          /---------------
%                ----------
%                ---------- 
% --------------/          \---------------
% \end{verbatim}

\begin{center}
  \begin{tikzpicture}[scale=0.5]

\foreach \y in {1.5,2,..., 4}{
	\pgfmathsetmacro{\yp}{\y-1}
	\draw [thick, accent-blue, <-] (7,-\yp) .. controls (3,-\yp) and (2,-\y) .. (-1,-\y) (7,-\yp) .. controls (10,-\yp) and (12,-\y) .. (15,-\y);
	\draw [thick, accent-blue, <-] (7,\yp) .. controls (3,\yp) and (2,\y) .. (-1,\y) (7,\yp) .. controls (10,\yp) and (12,\y) .. (15,\y);
}
\end{tikzpicture}
\end{center}
We find that $\vec{B}_0$ is more intense at the pinch because
$\vec{J}_0$ is larger. Hence the plasma is expelled from the pinched
region, drawing the current density up, and making the problem
worse. The plasma is described as pinching off.

\section{Plasmas with Collision}
\label{sec:plasm-with-coll}

Consider a binary collision of two plasma particles, where the Coulomb
force between the particles is
\begin{equation}
  \label{eq:15}
  F = \frac{q_1 q_2}{4 \pi \epsilon_0 r^2}
\end{equation}
For simplicity, we restrict the interaction to an electron and a heavy ion.
\begin{figure}
  \centering
  \begin{tikzpicture}[]
	% Draw the proton
	\fill [muted-blue] (0,0) circle (0.5) node [midway, white] {+};
	% Draw the electron
	\fill [muted-orange] (-4,1) circle (0.2) node [white] {-};
	
	\draw [->, thick] (-3.7,1) -- (0,1)  arc (90:40:1) -- ++(130:-2);

	\draw [help lines] (0,1) -- (4,1) (1,0)--(4,0);

	\draw [<->] (3,1) -- (3,0)  node [right, midway] {$b = r \sin(\theta)$};
\end{tikzpicture}
  \caption{The Interaction between an ion and electron.}
  \label{fig:electronioninter}
\end{figure}
Consider the change of $v_{\perp}$, with a massive ion, $m_{\rm ion}
\gg m_{\rm e}$. The change of perpendicular momentum from the equation
of motion can be written
\begin{equation}
  \label{eq:16}
  m_{\rm e} v_{\perp} = \int_{-\infty}^{+\infty} F \dd{t} \sim F \Delta t
\end{equation}
So, we can approximate,
\[ m_{\rm e} v_{\perp} \approx \frac{q_{\rm e} q_{\rm i}}{4 \pi
  \epsilon_0 r^{2} } \frac{r}{v} = \frac{q_{\rm e} q_{\rm i}}{4 \pi
  \epsilon_0 r v }  \] 
For large-angle collisions (where the deflection
is close to $90^{\circ}$), the change of $m_{\rm e} v_{\perp}$ is of
the order of $mv$ itself, so,
\[ m_{\rm e} v_{\perp} \approx m_{\rm e} v = \frac{q_{\rm e} q_{\rm i}}{4 \pi
   \epsilon_0 r v }\]
and $r \approx b$, so we can estimate $b$ as 
\[ b = \frac{q_{\rm e} q_{\rm i}}{4 \pi \epsilon_0 v^2 m_{\rm e}} \]

The interaction between two particles can be described using the
interaction cross-section, $\sigma$, where
\[ \sigma = \pi b^2 \] with $b$ being the area of the disc. We simply
assume the interaction is happening for impact parameters
\[ b \ll \frac{q_{\rm e} q_{\rm i}}{4 \pi \epsilon_0 v^2 m_{\rm e}} \]
and no interaction for large $b$.

For this interaction the cross-section is 
\begin{equation}
  \label{eq:17}
  \sigma_{\rm ei} = \pi b^2 = \pi \frac{q_{\rm e}^2 q_{\rm i}^2}{(4 \pi \epsilon_0)^2 (m_{\rm e} v)^2}
\end{equation}
The collision frequency is then
\[ \nu_{\rm ei} = n \sigma_{\rm ei} v \] Then, the e-i collision
frequency for a $z=1$ plasma, so that $n_{\rm i} \approx n_{\rm e}$,
and $q_{\rm i} = q_{\rm e} = e$ is
\begin{equation}
  \label{eq:18}
  \nu_{\rm ei} = n_{\rm e} \sigma_{\rm ei} v \approx \frac{\pi n_{\rm e} e^4}{(4 \pi \epsilon_0)^2 m_{\rm e}^2 v^3} \sim \frac{n_{\rm e}}{v^3}
\end{equation}
For the average electron velocity with thermal energy,
\[ k_{\rm B} T_{\rm e} = \half m_{\rm e} v_{\rm te}^2 \]
thus
\[ v_{\rm te} = \qty( \frac{2 k_{\rm B} T_{\rm e}}{m}
)^{-\frac{1}{2}} \]
\begin{equation}
  \label{eq:19}
  \nu_{\rm ei} \approx \frac{\sqrt{2}}{64 \pi} \frac{\omega_{\rm pe}^4}{n_{\rm e}} \qty( \frac{k_{\rm B} T_{\rm e}}{m})^{-\frac{3}{2}} = \frac{\pi n_{\rm e} e^4}{2^{\frac{3}{2}} (4 \pi \epsilon_0)^2 m_{\rm e}^2 \qty( \frac{k_{\rm B} T_{\rm e}}{m_{\rm e}})^{\frac{3}{2}}}
\end{equation}
So we conclude
\[ \nu_{\rm ei} \sim nT^{- \frac{3}{2}} \] This is a rough estimate,
there are many small-scale collsions in plasma, and a more rigorous
estimate indicates
\begin{equation}
  \label{eq:20}
  \nu_{\rm ei} = \frac{4 \sqrt{2}}{3 \sqrt{\pi}} \frac{\pi n_{\rm e} e^4 (\log(\Lambda)}{(4 \pi \epsilon_0)^2 m_{\rm e}^{\half} (k_{\rm B}T_{\rm e})^{\frac{3}{2}}} \approx \omega_{\rm pe} \frac{\log(\Lambda)}{\Lambda}
\end{equation}
Where $\Lambda$ is the number of electrons in the Debye Sphere,
$\Lambda \sim n_{\rm e} \lambda^3_{\rm De}$.  $\log(\Lambda)$ is the
Coulomb logarithm, and is normally assumed to be constant, with a
value in the range $\log(\Lambda) \in [10,30]$.

\subsection{Mean Free Path}
\label{sec:mean-free-path}

Let us estimate the mean free path of an electron in a plasma;
\[ \lambda = \frac{v_{\rho}}{\nu_{\rm ei}} \sim \frac{\omega_{\rm pe}
  \lambda_{\rho \rm e}}{\nu_{\rm ei}} \approx \qty( \frac{\omega_{\rm
    pe}}{\nu_{\rm ei}}) \lambda_{\rm De} \]

\begin{example}{\em The mean free path of the solar corona.}
  In the plasma composing the solar corona,
\[ n_{\rm e} \sim 1 \e{15} \meter^{-3} , \qquad k_{\rm B} T_{\rm e} \sim 1\e{2} \electronvolt \]
Then
\[ \nu_{\rm ei} \approx \frac{5\e{-11} \times 10^{15}}{10^3} \approx 50 \second^{-1} \]
and since
\[ \omega_{\rm pe} \approx 2 \pi \nu_{\rm pe}, \qquad \nu_{\rm pe} = 9(n_e)^{\half} \]
we find
\[ \omega_{\rm pe} \sim 2 \pi \times 9 \times 3.2\e{3} \times 10^4 = 2
\pi \times 3\e{8} \second^{-1} \gg \nu_{\rm ei} \] Hence the mean free
path is much larger than the Debye length.
\end{example}

\subsection{Collision Equilibration Times}
\label{sec:coll-equil-times}

For electron-ion collisions in plasma 
\[ \nu_{\rm ei} \propto \frac{n}{m_{\rm e}^{\half} T_{\rm
    e}^{\frac{3}{2}}} \] We define the quantity $\tau_{\rm ei} :=
\frac{1}{\nu_{\rm ei}}$ which is the time between collisions, or the
mean free time.
For the frequency of electron-electron collisions (taking into account the finite mass of the scattering particle, and replacing it with $m_{\rm e}$) we have a factor of two, and so,
\[ \nu_{\rm ee} \approx \nu_{ei} \]
For ion-ion collisions the $m_{\rm e}$ must become $m_{\rm i}$, and so
\[ \nu_{\rm ii} \approx \qty( \frac{m_{\rm e}}{m_{\rm i}} )^{\half}
\nu_{\rm ee} \] For ion-electron collsions the transformation to the
centre-of-mass frame introduces a factor of $\frac{m_{\rm e}}{m_{\rm
    i}}$, hence 
\[ \nu_{\rm ie} \approx \frac{m_{\rm e}}{m_{\rm i}} \nu_{\rm ee} \]

If $T_{\rm e} \neq T_{\rm i}$, there will be an exchange of
temperature caused by the collisions, and the timescales of the interactions are
\[ \tau_{\rm ee}^{\rm E} : \tau_{\rm ii}^{\rm E} : \tau_{\rm ei}^{\rm
  E} \sim 1 : \qty( \frac{m_{\rm i}}{m_{\rm e}} )^{\half} :
\frac{m_{\rm i}}{m_{\rm e}} \] It is worth noting that $\nu_{\rm ei}$
is not the rate at which equilibrium is established between electrons
and ions, but is instead the rate of momentum transfer from electrons
to ions, and not the rate of energy transfer between them. The
relaxation time for electorn-ion equiilibrium is given by ion-electron
collisions, and $\frac{m_{\rm e}}{m_{\rm i}} \nu_{\rm ee}$. For a Hydrogen plasma
\[ \frac{m_{\rm i}}{m_{\rm e}} = 1836 \]
so
\begin{equation}
  \label{eq:21}
  \tau_{\rm ee} : \tau_{\rm ii} : \tau_{\rm ei}^{\rm E} \sim 1 : 43 : 1836
\end{equation}

\subsection{Resitivity and Collisions}
\label{sec:resit-coll}

Consider an unmagnetised quasineutral plasma with $n_{\rm i} \approx
n_{\rm e}$ of electrons and ions both with charge $q = e$. In response
to an applied electric field, $\vec{E}$, a current will flow in the
plasma. The current density will be 
\begin{equation}
  \label{eq:22}
  \vec{\jmath} = n_{\rm i} e \vec{v}_{\rm i} - n_{\rm e} e \vec{v}_{\rm e}
\end{equation}
Electrons have a much smaller mass than the ions, so the plasma
current is predominantly carried by the electrons, hence,
\begin{equation}
  \label{eq:23}
  m_{\rm e} n_{\rm e} \dv{\vec{v}_{\rm e}}{t} = -e n_{\rm e} \vec{E} + m_{\rm e} n_{\rm e} (\vec{v}_{\rm i} - \vec{v}_{\rm e}) \nu_{\rm ei}
\end{equation}
In a steady state there is no change with time, so
\[ \vec{E} = \frac{m_{\rm e} n_{\rm e} (\vec{v}_{\rm i} - \vec{v}_{\rm e}) \nu_{\rm ei}}{e n_{\rm e}} = \frac{e n_{\rm e} (\vec{v}_{\rm i}-\vec{v}_{\rm e} ) m_{\rm e} \nu_{\rm ei}}{e^2 n_{\rm e}} = \frac{m_{\rm e} \nu_{\rm ei}}{e^2 n_{\rm e}} \vec{\jmath} \]
According to Ohm's law,
\begin{equation}
  \label{eq:24}
  \vec{E} = \rho \vec{\jmath}
\end{equation}
where $\rho$ is the resistivity,
\[ \rho = \frac{m_{\rm e}}{e^2 n_{\rm e}} \nu_{\rm ei} =
\frac{1}{\sigma} \] with $\sigma$ the conductivity.  Conservation of
momentum prevents electron-electron collisions contributing to the
resisitivity, and so
\[ \rho = \frac{m_{\rm e}}{e^2 n_{\rm e}} \frac{\pi n_{\rm e} e^4
  \log(\Lambda)}{\qty(4 \pi \epsilon_0)^2 m_{\rm e}^{\half}
  \qty(k_{\rm B} T)^{\frac{3}{2}}} = \frac{\pi m_{\rm e}^{\half} e^2
  \log(\Lambda)}{(4 \pi \epsilon_0)^2 (k_{\rm B}T)^{\frac{3}{2}}} \]
Again, it is worth noting that resistivity is independent of density,
and decreases with growing temperature.

\subsection{Diffusion of Particles}
\label{sec:diffusion-particles}

The fluid equation of motion including collisions for electrons is
\begin{equation}
  \label{eq:25}
  m_{\rm e} n_{\rm e} \dv{\vec{v}}{t} = q n_{\rm e} \vec{E} - \nabla p - m_{\rm e} n_{\rm e} \nu_{\rm ei} \vec{v}
\end{equation}
for pressure $p$, and assuming $n_{\rm e} \approx n_{\rm i} = n$, and
$\nu_{\rm ei}$ is constant. Considering a steady state,
\[ q n_{\rm e} \vec{E} - \nabla p - m_{\rm e} n_{\rm e} \nu_{\rm ei}
\vec{v} = 0 \]
and so
\begin{align*}
  \vec{v} &= \frac{1}{m_{\rm p} m_{\rm e} \nu_{\rm ei}} \qty( q n \vec{E} - k_{\rm B} T \nabla n_{\rm e} ) \\
&= \frac{q}{m_{\rm e} \nu_{\rm ei}} \vec{E} - \frac{k_{\rm B} T}{m_{\rm e} \nu_{\rm ei}} \frac{\nabla n}{n} \\
&= \mu \vec{E} - D \frac{\nabla n}{n}
\end{align*}
where $\mu$ is the mobility coefficient, and $D$ the diffusion
coefficient. These differ for each species.

The flux, $\vec{\Gamma}$, is
\[ \vec{\Gamma} = n \cdot \vec{v} = \pm \mu n\vec{E} - D \nabla n \]
if either there is no $\vec{E}$-field, or the particles are uncharged we find Fick's Law,
\begin{equation}
  \label{eq:26}
  \vec{\Gamma} = -D \nabla n
\end{equation}
From the continuity equation,
\[ \pdv{n}{t} + \nabla \cdot \vec{\Gamma} = 0 \]
we can construct the diffusion equation,
\begin{equation}
  \label{eq:27}
  \pdv{n}{t} - \nabla D \nabla n = 0
\end{equation}

\subsection{Ambipolar Diffusion}
\label{sec:ambipolar-diffusion}

A plasma should be quasi-neutral, so diffusion of electrons and ions
should adjust to some degree to preserve quasineutrality. The
fast-moving electrons have higher thermal velocities and tend to leave
a plasma first. The positive charge is then left behind, and an
electric field is setup to retard the loss of electrons, and
accelerate the loss of ions. We let $\vec{\Gamma}_{\rm e} =
\vec{\Gamma}_{\rm i} = \vec{\Gamma}$, so
\[ \vec{\Gamma} = \mu_{\rm i} n \vec{E} - D_{\rm i} \nabla n =
\mu_{\rm e} n \vec{E} - D_{\rm e} \nabla n \]
and solving for $\vec{E}$ we find
\[ \vec{E} = \frac{D_{\rm i} - D_{\rm e}}{\mu_{\rm i} + \mu_{\rm e}}
\frac{\nabla n}{n} \]
and the total flux is
\[ \vec{\Gamma} = \mu_{\rm i} n \frac{D_{\rm i} - D_{\rm e}}{\mu_{\rm
    i}+\mu_{\rm e}} \frac{\nabla n}{n} - D_{\rm i} \nabla n = -
\frac{\mu_{\rm i} D_{\rm e} + \mu_{\rm e} D_{\rm i}}{\mu_{\rm i} +
  \mu_{\rm e}} \nabla n \]
Which is just Fick's law again, with an additional coefficient,
\[ D_{\rm a} = \frac{\mu_{\rm i} D_{\rm e} + \mu_{\rm e}D_{\rm
    i}}{\mu_{\rm i} + \mu_{\rm e}} \] which is the ambipolar diffusion
coefficient.

\section{Kinetic Theory of Plasmas}
\label{sec:kinet-theory-plasm}

There are some phenomena which neither MHD nor single-particle
descriptions of a plasma can describe. For these situations we need to
consider the velocity distribution, $f(\vec{v})$ of the plasma.  In
fluid theory the independent variables are functions of $\vec{r}$ and
$t$ only, which is because the velocity distribution is taken to be
Maxwellian everywhere, so can be uniquely speciied by the temperature,
$T$, and the number density, $n(\vec{r}, t)$,
\[ n(\vec{r}, t) = \int f(\vec{r}, \vec{v}, t) \d[3]{V} = \int
f(\vec{r}, \vec{v}, t) \dd[3]{V} \] If $f$ is correctly normalised it
describes the probbaility of finding a particle in the range $\vec{r}
\in (\vec{r}, \vec{r}+\dd{\vec{r}})$ and $\vec{v} \in (\vec{v},
\vec{v}+\dd{\vec{v}})$.  $f$ is a function in seven variables, and, if
it is a Maxwellian distribution, it has the form,
\begin{equation}
  \label{eq:28}
  f(\vec{r}, \vec{v}, t) = n(\vec{r}, t) \frac{m}{(2 \pi k T)^{\frac{3}{2}}} \exp( - \frac{v^2}{v_{\rm th}^2})
\end{equation}
for 
\[ v_{\rm th} = \qty(\frac{2kT}{m})^{\half} \]

Ignoring collisions, and assuming the plasma to be a closed system,
with no sources or sinks of particles, the function will obey the
Liouville theorem, so
\[ \dv{f}{t} = 0 \] for the time derivative along a trajectory in
$(\vec{r}, \vec{v})$-phase space, so
\begin{align*}
  \dv{f}{t} = \pdv{f}{t} + \dv{\vec{r}}{t} \pdv{f}{\vec{r}} + \dv{\vec{v}}{t} \pdv{f}{\vec{v}} &= 0 \\
\pdv{f}{t} + \vec{v} \pdv{f}{\vec{r}} + \frac{\vec{F}}{m} \pdv{f}{\vec{v}} &= 0
\end{align*}
which is the kinetic equation for the plasma. If the force is entirely
electromagnetic the equation takes the form
\begin{equation}
  \label{eq:29}
  \pdv{f}{t} + \vec{v} \pdv{f}{\vec{r}} + \frac{q}{m} \qty( \vec{E} + \vec{v} \times \vec{B}) \pdv{f}{\vec{v}} = 0
\end{equation}
which is the Vlasov equation. This should be completed with the system
of Maxwell's equations, as $\vec{E}$ and $\vec{B}$ are the average
values of the electric and magnetic fields from particles in the
plasma.

If there are collisions in the plasma, $\dv{f}{t}\neq 0$, and, using
the collision integral,
\begin{equation}
  \label{eq:30}
  \pdv{f}{t} + \vec{v} \pdv{f}{\vec{r}} + \frac{\vec{F}}{m} \pdv{f}{\vec{v}} = \qty( \pdv{f}{t} )_{\rm coll} 
\end{equation}
which is the Boltzmann equation. The kinetic equation can be modified
to include sources or sinks of particles by adding terms on the right
hand side.

The collision term can sometimes be approximated as
\[ \qty( \pdv{f}{t} )_{\rm coll} = \frac{f_{\rm eq} - f}{\tau} \]
where $f_{\rm eq}$ is the equilibrium function, and $\tau$ is the
collision time. This is the Krook collision term.

\section{Plasma waves and Landau Damping}
\label{sec:plasma-waves-landau}

As an illustration of the use of the Vlasov equation, we consider the
electron plasma oscillations in a uniform plasma with no applied
magnetic or electric fields. Consider a first order perturbation,

\[ f(\vec{r}, \vec{v}, t) = f_0(\vec{r}, \vec{v}, t) + f_1(\vec{r},
\vec{v}, t) \]
the first order Vlasov equation for electrons is 
\[ \pdv{f_1}{t} + \vec{v} \nabla f_1 = \frac{e}{m} \vec{E}_1
\pdv{f_0}{\vec{v}} =0 \] where $E_1$ is a perturbation of the electric
field. Using Poisson's equation ($\epsilon_0 \nabla \vec{E} = \rho$),
\[ \epsilon_0 \nabla \vec{E}_1 = -e n_1 = -e \int f_1 \dd[3]{V} \] As
before assume the ions are massive and immobile, and assume the waves
in the plasma are plane waves in the $x$-direction. Then,
\[ f_1 = f_0 \exp(- i \omega t + i k x) \]
\[ E_1 = E_x \exp( - i \omega t + i k x) \]
Then we can write
\[ - i \omega f_1 + i k v_x f_1 = \frac{e}{m} E_x \pdv{f_0}{v_x} \]
from the Vlasov equation, and
\[ \epsilon_0 i k E_x = - \int f_1 \dd[3]{V} \]
from Poisson's equation. Combining the two,
\begin{align*}
  1 & = - \frac{e^2 }{l m \epsilon_0} \int \frac{\pdv{f_0}{v_x}}{\omega- k v_x} \dd[3]{V} \\
  &= - \frac{e^2}{k m \epsilon_0} \int_{-\infty}^{\infty} \dd{v_z}
  \int_{-\infty}^{\infty} \dd{v_y} \int \frac{\pdv{f_0}{v_x}}{\omega-
    k v_x} \dd{v_x}
\end{align*}
If $f_0$ is Maxwellian the integration over $v_y$ and $v_z$ can be carried out, and
\[ f_0(v_x) = n_0 \frac{m}{(2 \pi k_{\rm B} T)^{\half}} \exp( -
\frac{m v_x^2}{2 k_{\rm B} T} ) \] Taking the normalised function
$\tilde{f} = \frac{f_0}{n_0}$,
\[ 1 = \frac{\omega_{\rm pe}^2}{k^2} \int_{-\infty}^{\infty}
\frac{\pdv{\tilde{f}_0}{v_x}}{v_x - \frac{\omega}{k}} \dd{v_x} \] This
integral is non-trivial to compute, due to the singularity at $v_x =
\frac{\omega}{k}$. Landau suggested (1946) letting $\omega \to \omega
+ i o$ for a small $o$, which makes the integral
\[ \int_{-\infty}^{\infty} \frac{y(z)}{z-io} \dd{z} = P
\int_{-\infty}^{\infty} \frac{f(z)}{z} \dd{z} + i \pi f(0)\] where $P$
denotes the Cauchy principle value of the integral. This can be written symbolically as
\[ \frac{1}{z-io} = P \frac{1}{z} + i \pi \delta(z) \]
and then, using the Landau rule, and letting $v := v_x$,
\begin{equation}
  \label{eq:31}
  1 = \frac{\omega_{\rm pe}^2}{k^2} \qty[ P \int_{-\infty}^{\infty} \frac{\pdv*{\tilde{f}_0}{v}}{v - \frac{\omega}{k}} + i\eval{\pi \pdv{\tilde{f}_0}{v} }_{v = \frac{\omega}{k}}]
\end{equation}

First, concentrate on the real part, where the integral can be
computed by integrating by parts,
\begin{align*} 
\int_{-\infty}^{\infty} \pdv{\tilde{f}_0}{v} \frac{\dd{v}}{v - \frac{\omega}{k}} 
&= \eval{\frac{\tilde{f}_0 \dd{v}}{(v - \frac{\omega}{k})^2}}_{-\infty}^{\infty} - \int - \frac{\tilde{f}_0 \dd{v}}{\qty(v - \frac{\omega}{k})^2} \\ 
&= \int_{-\infty}^{\infty} \frac{\tilde{f}_0 \dd{v}}{\qty( v - \frac{\omega}{k})^2}
\end{align*}
We can assume $\frac{\omega}{k} \gg v$ (i.e.\ large phase velocities),
and so we can expand $(v - \frac{\omega}{k})^{-2}$,

\begin{align*} (v - \frac{\omega}{k})^{-2} &= \qty( \frac{\omega}{k} )^{-2} \qty( 1 - \frac{v k}{\omega})^{-2} \\
&= \qty( \frac{\omega}{k} )^{-2} \qty( 1 + \frac{2 v k}{\omega} + \frac{3(vk)^2}{\omega^2} + \cdots )
\end{align*}
And now, using the expansion for the integral,
\begin{align*}
  \int_{-\infty}^{\infty} \frac{\tilde{f}_0 \dd{v}}{\qty( v - \frac{\omega}{k})^2} &= \qty( \frac{\omega}{k} )^{-2} \int_{- \infty}^{\infty} \qty( 1 + \frac{2 v k}{\omega} + \frac{3(vk)^2}{\omega^2} + \cdots ) \tilde{f}_0 \dd{v}
\end{align*}
The odd terms in $v$ will vanish, and 
\[ \int_{-\infty}^{\infty} v^2 \tilde{f}_0 \dd{v} \]
is just an average, so assuming $\tilde{f}_0$ is Maxwellian,
\[ \int_{-\infty}^{\infty} v^2 \tilde{f}_0 \dd{v} = \frac{k_{\rm B}T_{\rm e}}{m}\]

Then we can write
\[ 1 = \frac{\omega_{\rm pe}^2}{k^2} \qty[ \qty(\frac{\omega}{k})^{-2} \qty(1+ \frac{3 k_{\rm B} T_{\rm e} k^2}{m \omega^2})] = \frac{\omega_{\rm pe}^2}{\omega^2} \qty( 1+ \frac{3 k_{\rm B} T_{\rm e} k^2}{m \omega^2} )\]
If the thermal correction is small we can replace $\omega^2$ with $\omega_{\rm pe}^2$ in the second term, so
\begin{equation}
  \label{eq:32}
  \omega^2(k) \approx \omega_{\rm pe}^2 + \frac{3 k_{\rm B} T_{\rm e}}{m} k^2
\end{equation}
This is the dispertion relation for Langmuir waves. The phase speed is 
\[ v_{\rm p} = \frac{\omega}{k} \approx \frac{\omega_{\rm pe}}{k} \]
and the group velocity is
\[ v_{\rm g} = \pdv{\omega}{k} \approx \frac{3 k_{\rm B} T_{\rm e}}{m}
k \] (assuming that $\omega(k) \approx \omega_{\rm pe} + \frac{3}{2}
\frac{k_{\rm B} T_{\rm e}}{m} \frac{k^2}{\omega_{\rm pe}}$). Then, the
group velocity is
\[ v_{\rm g} \approx 3 \frac{k_{\rm B}T_{\rm e}}{m}
\frac{k}{\omega_{\rm pe}} \approx 3 \frac{k_{\rm B} T_{\rm e}}{m}
\frac{1}{v_{\rm p}} \approx 3 \frac{v_{\rm Te}^2}{v_{\rm p}} \]

Finally, the imaginary part. For simplicity ignore the thermal
correction, so $\omega(k) \approx \omega_{\rm pe}$, then
\[ 1 = \frac{\omega_{\rm pe}^2}{\omega^2} + i \pi \frac{\omega_{\rm
    pe}^2}{k^2} \eval{\pdv{\tilde{f}_0}{v}}_{v=\frac{\omega}{k}} \]
and so
\[ \omega^2 = \omega_{\rm pe} \qty( 1 - i \pi \frac{\omega_{\rm
    pe}^2}{k^2} \eval{ \pdv{\tilde{f}_0}{v} }_{v=\frac{\omega}{k}}
)^{-1} \]
Then, assuming that the imaginary part is small,
\begin{equation}
  \label{eq:33}
  \omega(k) = \omega_{\rm pe} + i \omega_{\rm pe} \frac{\pi}{2} \frac{\omega_{\rm pe}^2}{k} \eval{\pdv{\tilde{f}_0}{v}}_{v = \frac{\omega}{k}}
\end{equation}
which is Landau damping.  Substituting the one-dimensional Maxwellian
distribution for $\tilde{f}$,
\begin{align*} 
\pdv{\tilde{f}}{v} &= \qty( \pi v_{\rm Th}^2 )^{- \half} \exp( - \frac{v^2}{v_{\rm Th}^2}) \qty( - \frac{2v}{v_{\rm Th}^2}) \\ &\approx - \frac{2v}{\sqrt{\pi} v_{\rm Th}^3} \exp( - \frac{v^2}{v_{\rm Th}^2} ) 
\end{align*}
The using the knowledge that $v= \frac{\omega}{k} \approx
\frac{\omega_{\rm pe}}{k}$,
\begin{align}
  \gamma_k = \Im(\omega) &= - \frac{\pi}{2} \frac{\omega_{\rm pe}^3}{k^2} \frac{2 \omega_{\rm pe}}{k \sqrt{\pi}} \frac{1}{v_{\rm Th}^3} \exp( - \frac{\omega_{\rm pe}^2}{k^2 v_{\rm Th}^2} ) \nonumber\\
&= - \sqrt{\pi} \omega_{\rm p} \qty( \frac{\omega_{\rm pe}}{k v_{\rm Th}})^3 \exp( - \frac{\omega_{\rm pe}}{k^2 v_{\rm Th}^2})
\end{align}
Which is Landau damping. A useful equation derived from this is then
\begin{equation}
  \label{eq:34}
  \Im\qty(\frac{\omega}{\omega_{\rm pe}}) \approx -0.22 \sqrt{\pi} \qty( \frac{\omega_{\rm pe}}{k v_{\rm Th}})^3 \exp(- \frac{1}{2k^2 \lambda^2_{\rm De}} )
\end{equation}
The discovery of wave damping without collisional dissipation has been described as ``probably the most astounding result of plasma physics.''

The damping is not reandomisation by collisions, but a resonant (i.e.\
phase velocity of waves is the same as the velocity of interacting
particles) $v=\frac{\omega}{k}$ transfer of energy from waves to
particles. It can be reversed if $\pdv*{\tilde{f}_0}{v}>0$.
\chapter{Plasma and Radiation}
\label{cha:plasma-radiation}

\section{Plasma Radiation}
\label{sec:plasma-radiation}


\begin{figure}
  \centering
  \begin{tikzpicture}[scale=0.4]
\begin{scope}[]
	\foreach \t in {0,30,...,360}{
          \coordinate (B\t) (0,0)+(\t:7);
		\draw [accent-blue, thick, ->](0,0) +(\t:7)  -- +(\t:10);
	}
	\fill [muted-cream,opacity=1] (0,0) circle (7);	
\end{scope}
\begin{scope}[]
	\fill [white] (0,0) circle (5);
	\foreach \t in {0,30,...,360}{
		\draw [accent-green, thick, ->] (2,0) -- (\t:5) coordinate (A\t);
	}
	
        \fill (0,0) node [fill=accent-blue, circle] {1};
        \fill (2,0) node [fill=accent-green, circle] {2};
\end{scope}
\begin{scope}[]
	\foreach \t in {0,30,...,360}{
          \pgfmathsetmacro{\back}{\t+30}
		\draw [accent-pink, thick, ->] (0,0)+(\t:5) -- +(\back:7);
	}
\end{scope}

\draw [<->, accent-red, ultra thick] (0,.6) -- (2,.6) node [above, midway] {$\Delta v \ t$};

\draw [<->, accent-red, ultra thick] (0,-.6) -- (-5,-.6) node [fill=white, midway] {$r = ct$};
\draw [<->, accent-red, ultra thick] (-170:5) -- (-170:7) node [below, yshift=-.1cm, midway] {$r^{\prime} = c \Delta t$};

\end{tikzpicture}
  \caption{The electric field surrounding an accelerating charge.}
  \label{fig:acceleratingcharge}
\end{figure}


Charged particles will emit electromagnetic radiation if they
accelerate during their motion. An example of accelerated motion in a
plasma is the gyromotion of particles; radiation from the gyromotion
is generally called synchotron radiation. Extagalactic jets, solar
flares, and supernova remnants all show this behaviour. \\To
understand how this radiation is produced, consider a charge, $q$ at a
time $t=0$, which is stationary at the origin of a laboratory rest
frame. The electric field of the charge can be visualised as lines
radiating from the charge. Now, let the particle accelerate to $\Delta
\vec{v}$ in a time $\Delta t$. We assume that $\abs{ \Delta \vec{v} }
\ll c$, so that relativistic corrections are small. After the
acceleration the field of the particle will be radial about its new
location. This field extends a distance of $r = ct$ from the new
location of the charge. The old field will continue to exist in the
region $r> c(t + \Delta t)$. In the region between these two there is
a thin shell, with thickness $c \Delta t$ where the field lines from
the before and after cases must join up. As a result there must be a
non-radial component of $\vec{E}$ in this region, and this constitutes
a propogating pulse of electromagnetic field. \\
The electric field lines are radial at $t=0$ and $t=T$, but they have
different origins. The lines joining the old and new field lines have
a non-radial electric field component, $\vec{E}_{\phi}$, in addition
to the radial component $\vec{E}_{r}$.

The radial component has the usual form,
\[ \vec{E}_r = \frac{q}{4 \pi \epsilon_0} \frac{\vec{r}}{r^2} \] while
the $\vec{E}_{\phi}$ component is given by the number of radial
field-lines per unit area in the direction of $\vec{\phi}$. From geometry arguments,
\begin{equation}
  \label{eq:9}
  \frac{E_{\phi}}{E_r} = \frac{\Delta v t \sin(\phi)}{c \Delta t}
\end{equation}
then, since $t = \frac{r}{c}$, $\frac{\Delta v}{\Delta t} = \ddot{r}$,
and substituting $\vec{E}_r$,
\begin{equation}
  \label{eq:10}
  E_{\phi} = \ddot{r} \sin(\phi) \frac{q}{4 \pi \epsilon_0} \frac{1}{c^2 r}
\end{equation}
$\vec{E}_{\phi}$ is known as the \emph{acceleration field}, and has a
strength varying with $\frac{1}{r}$, in contrast to the radial field
which varies with $\frac{1}{r}$. This ``kink'' is an outward moving
pulse of electromagnetic radiation. The power per unit area per second
is given by the Poynting vector,
\[ \vec{S} = \frac{\vec{E} \times \vec{B}}{\mu_0} = c \epsilon_0 E^2
\hat{r} \]
The total energy radiated per second then becomes 
\[ P(t) = \int \vec{S} \cdot \dd{\vec{A}} = \int \vec{S} r^2
\dd{\vec{\Omega}} \]
Recalling that for a sphere, $\dd{\Omega} = 2 \pi \sin(\phi) \dd{\phi}$,
\begin{equation}
  \label{eq:11}
  P(t) = \frac{q^2 |\ddot{r}|^2}{6 \pi \epsilon_0 c^3}
\end{equation}
which is Larmour's formula. This is a general expression for the
radiated power from an accelerated charge, and can be applied to our
specific case of a particle gyrating in an electric field.

\subsection{Cyclotron Radiation}
\label{sec:cyclotron-radiation}

Consider the rest frame of a charge $q$, called $S^{\prime}$, and a
lab frame $S$. 
\[ \text{In the $S$ frame} \quad m \dv{\vec{v}}{t} = q \vec{v} \times \vec{B} \]
\[ \text{In the $S^{\prime}$ frame} \quad m
\dv{\vec{v}^{\prime}}{t^{\prime}} = q \vec{B}^{\prime} \] The electric
field $E^{\prime} = v \gamma B \sin(\theta)$ (with $\theta$ the
electron pitch angle) due to the relativistic transformations of
$\vec{E}$ and $\vec{B}$.
Using Larmour's formula,
\begin{align*}
  P(t)^{\prime} &= \frac{q^2 \abs{\dot{v}^{\prime}}^2}{6 \pi
    \epsilon_0 c^3}\\ &= \frac{q^2}{6 \pi \epsilon_0 c^3}
  \frac{q^2}{m^2} \abs{\vec{B}^{\prime}}^2 \\ &= \frac{q^4}{6 \pi
    \epsilon_0 c^3 m^2} (v \gamma B)^2 \sin[2](\theta)
\end{align*}
This gives the power radiated in the rest frame of the electron. Power
is Lorentz invariant ($P = P^{\prime}$), so the power in the lab frame
must be the same. The total power radiated in the lab frame from a
particle with pitch anfgle $\theta$ is
\[ P = \frac{q^4 B^2}{6 \pi \epsilon_0 c m^2} \gamma^2 \beta
\sin[2](\theta) \] for $\beta = \frac{v}{c}$.
and simplifying,
\[ P = 2 \sigma_{\rm T} c U_{\rm mag} \gamma^2 \sin[2](\theta) \] with
$\sigma_{\rm T}$ the Thomson cross-section, $U_{\rm mag} =
\frac{B^2}{2 \mu_0}$.

The distribution of the radiation about the moving charge is worth
considering;
\begin{equation}
  \label{eq:12}
  \dv{P}{\Omega} = c \epsilon_0 \frac{q^2}{(4 \pi \epsilon_0)^2} \frac{(\ddot{r})^2}{c^4} \sin[2](\phi)
\end{equation}
this is a dipole emission pattern.

If the particle is relativistic we must consider the effect on the
cyclotron frequency. The relativistic frequency will be
\[ \omega_{\rm r} = \frac{\omega_{\rm c}}{\gamma} =
\frac{\abs{q}B}{\gamma m_0} \] where $m_0$ is the rest-mass of the
particle. This can be decomposed into a series of harmonics,
$\omega_n$,
\[ \omega_n = \frac{n \omega_{\rm r}}{\qty(1 - \beta_{\parallel}
  \cos(\theta) )} \] As a result, the spectrum of highly relativistic
electrons will be distinctly different from the delta-function peak we
expect from a non-relativistic charge. A distribution of electrons
will in fact display a power-law spectrum. As the charge becomes more
relativistic the dipole shape of the radiation is deformed, and an
effect known as ``relativistic beaming'' will be observed, with the
output of radiation becoming more and more focussed.

\section{Faraday Rotation in a Cold Plasma}
\label{sec:faraday}

Parallel to a uniform magnetic field, $\vec{B}_0$ there are two
electromagnetic modes (cold plasma modes), where
\[ n^2 = R \qquad n^2 = L \quad L \neq R\] thus different
polarisations have different phase speeds---we can exploit this as a
remote diagnostic of the plasma conditions.\\
Consider two circuarly polarised modes, $n^2 = \{R, L \}$, then
\[ n^2 = \frac{k^2 c^2}{\omega^2} \] Thus it is possible to convert
between $n$ and $\frac{\omega}{k}$. A superposition of the circuarly
polarised waves produces an evolution along the propogation ray of the
net $\vec{E}$ polarisation direction.

\begin{tikzpicture}[scale=0.5]

\draw [ultra thick, ->] (0,-4.4) -- (7,-4.4) node [midway, below] {$z$};

\begin{scope}[]
	\filldraw [thick, draw=muted-blue,  fill=muted-blue!10] (0,0) circle (2);
	\draw [ultra thick, accent-red, <->] (0,-2) -- (0,2) node [midway, right, black] {$\vec{E}$};
	\draw (2,4) node [text width=3cm, left, text ragged] {At $z=0$ $R$ and $L$ are synchronised, producing a vertical polarisation.};
	\filldraw [thick, draw=black,  fill=muted-green!10] (-1,-3) circle (.8) node {$R$};
	\filldraw [thick, draw=black,  fill=muted-orange!10] (1,-3) circle (.8) node {$L$};
	\fill (-1,-2.2) circle (0.1); \fill (1,-2.2) circle (0.1);
	\draw [thick, ->] (-1,-2) arc (90:135:1);
	\draw [thick, ->>] (1,-2) arc (270:225:-1);
\end{scope}

\begin{scope}[xshift = 200]
	\draw (0,4) node [text width=3cm, text ragged] {As $z$ increases, $R$ and $L$ the faster rotation of the $L$ mode causes the polarisation to become diagonal.};
	\filldraw [thick, draw=muted-blue,  fill=muted-blue!10] (0,0) circle (2);
	\draw [ultra thick, accent-red, <->] [rotate around={-30:(0,0)}](0,-2) -- (0,2) node [midway, right, black] {$\vec{E}$};

	\filldraw [thick, draw=black,  fill=muted-green!10] (-1,-3) circle (.8) node {$R$};
	\filldraw [thick, draw=black,  fill=muted-orange!10] (1,-3) circle (.8) node {$L$};
	\fill [rotate around={30:(-1,-3)}](-1,-2.2) circle (0.1); 
	\fill [rotate around={-60:(1,-3)}] (1,-2.2) circle (0.1);
	\draw [thick, ->] [rotate around={30:(-1,-3)}] (-1,-2) arc (90:135:1);
	\draw [thick, ->>] [rotate around={-60:(1,-3)}] (1,-2) arc (270:225:-1);
\end{scope}

\end{tikzpicture}

Suppose the polarisation vector $\vec{E}$ shifts by an angle
$\dd{\theta}$ as a result of the varying phase interference of the
superposed $R$ and $L$ modes. Then,
\[ \dd{\theta} = \frac{1}{2} (k_L - k_R) \dd{z} \]

Considering the special case of an electron plasma, with $\omega \gg
\omega_{\rm c_e}$, we can see that $n^2 = \{R,L\}$ are
\begin{align*}
  R & = 1 - \frac{\omega_{\rm p}^2}{\omega(\omega-\omega_{\rm c_e})} \\
  L & = 1 - \frac{\omega_{\rm p}^2}{\omega(\omega+\omega_{\rm c_e})}
\end{align*}
with $\omega_{\rm c_+} \to 0$, as $m_+ \to \infty$.
Thus
\begin{align*}
  \eval{n}_{\rm RCP} &\approx 1 - \frac{\omega_{\rm p}^2}{2 \omega (\omega-\omega_{\rm c_e})} \\
  \eval{n}_{\rm LCP} &\approx 1 - \frac{\omega_{\rm p}^2}{2 \omega(\omega+\omega_{\rm c_e})}
\end{align*}
Then
\begin{align*}
  (n_R - n_L) &= \frac{\omega_{\rm p}^2}{2 \omega} \qty[ \frac{1}{\omega - \omega_{\rm c_e}} - \frac{1}{\omega(\omega+\omega_{\rm c_e})}]\\
&= \frac{\omega_{\rm p}^2}{2 \omega} \frac{\omega + \omega_{\rm c_e}-(\omega-\omega_{\rm c_e})}{\omega^2 - \omega_{\rm c_e}^2} \\
&= \frac{\omega_{\rm p}^2 \omega_{\rm c_e}}{\omega(\omega^2-\omega_{\rm c_e}^2)} \\
&\approx \frac{\omega_{\rm p}^2 \omega_{\rm c_e}}{\omega^3} \\
\end{align*}
Hence,
\begin{align*}
  \dd{\theta} &= \half \qty| (k_R - k_L)| \dd{z} \\
& \approx \half \int_0^2 \frac{\omega_{\rm p}^2 \omega_{\rm c_e} \dd{z}}{c \omega^2} \\
&= \half \int_0^2 \frac{ \frac{ne^2}{m_{\rm e}^2} \frac{eB}{m_{\rm e}}}{ c \omega^2} \dd{z}\\
&= \frac{e^2}{2 \epsilon_0 m_{\rm e}^2 c \omega^2} \int_0^2 nB \dd{z}
\end{align*}
We could allow $n$ and $B$ to vary slowly, i.e.\ with the
gradient-scale length of $n,B$ much greater than the wavelength of the
circular poalrisation modes, and still have a Faraday rotation result
which is reasonably slow. \\
When observing the wave we can only make a detection at one point in
the propogation path, but using the frequency dependence we can detect
polarisation variation by measuring the polarisation at multiple
frequencies.\\
Using Faraday rotation allows the measurement of $\int nB \dd{z}$
along the line-of-sight path as a function of frequency. Suppose we
have a system with a plasma between the observer and a multispectral
event. We set off a flash which travels through the plasma, and
observe the radiation on the far side. Plasma slows different
frequencies to a different degree, so the time elapsed between
different frequencies allows the line-of-sight integrated number
density to be found, which in turn allows the integrated line-of-sight
magnetic field to be found, which can be used to make maps of magnetic
fields.

We can generalise this result, with
\begin{equation}
  \label{eq:faradayangle}
  \theta_{\rm F} = \frac{1}{\omega^2} \frac{e^3}{2 \epsilon_0 m_{\rm e}^2 c} \int_0^z n B \cos \theta \dd{z'}
\end{equation}
with $\theta_{\rm F}$ being the angle between the full magnetic field
direction and the line-of-sight direction.

For a cold plasma we have the dispertion relation in equation
(\ref{eq:dispertion}), for a wave with $\omega \gg \omega_{\rm p}$ and
$\omega \gg \omega_{\rm ci}$,
\begin{align*}
  S &= \half \qty( 1 + \frac{\omega_{\rm p}^2}{(\omega + \omega_{\rm ci})(\omega_{\rm ei}-\omega)} + 1 - \frac{\omega_{\rm p}^2}{(\omega - \omega_{\rm ei})(\omega+\omega_{\rm ce})}) \\
&= \half \qty( 2+ \omega_{\rm p}^2 \frac{(\omega-\omega_{\rm ci})(\omega+\omega_{\rm ce}) - (\omega+\omega_{\rm ci})(\omega_{\rm ce}-\omega)}{(\omega^2-\omega_{\rm ci}^2)(\omega_{\rm ce}^2 - \omega^2)}) \\
&= \half \qty( 2+ \omega_{\rm p}^2  \frac{\omega \qty(\omega+\omega_{\rm ce} - (\omega_{\rm ce}-\omega))}{\omega^2(\omega_{\rm ce}^2 - \omega^2)})\\
&= \half \qty( 2+ \omega_{\rm p}^2 \frac{2 \omega^2}{\omega^2 \qty(\omega_{\rm ce}^2 - \omega^2)} )\\
&= \half \qty( 2+ \omega_{\rm p}^2 \frac{2 \omega^2}{\omega^2 \qty(\omega^2_{\rm ce} - \omega^2)}) \\
&= 1 + \frac{\omega_{\rm p}^2}{\qty( \omega_{\rm ce}^2 - \omega^2)}
\end{align*}
Now introducing $\Gamma_{\rm p} := \frac{\omega_{\rm
    p}^2}{\omega^2_{\rm ce} - \omega^2}$,
finally,
\begin{align*}
  S &= \half (R+L) = 1+r_{\rm p}\\
D &= \half (R-L) = r_{\rm p} \frac{\omega_{\rm ce}}{\omega}\\
P &= 1 - \frac{\omega_{\rm p}^2}{\omega^2} \approx S
\end{align*}
In the limit $\omega \gg \omega_{\rm ce}$.  The general cold plasma
dispertion relation is equation (\ref{eq:dispertionrelation}), which
can be solved for $n^2$,
\begin{align*} 
n^2 &= \frac{2 PS +(RL-PS)(\sin[2](\theta)) \pm \sqrt{(RL-PS)^2 \sin[4](\theta)} }{{2 \qty[ P+(S-P) \sin[2](\theta)]}} \\& \quad+ \frac{{4 P^2D^2 \cos[2](\theta)}}{2 \qty[ P+(S-P) \sin[2](\theta)]} 
\end{align*}
In the limit $\omega \gg \omega_{\rm ce}$, $P \approx S$ and 
\[ RL - PS \approx  RL -S^2 = -D^2 \]
hence
\begin{align*}
  n_{\pm}^2 &= \frac{2 S^2 - D^2 \sin[2](\theta) \pm \sqrt{D^4 \sin[4](\theta) + 4 S^2 D^2 \cos[2](\theta)}}{2S}
\end{align*}
where the $\pm$ corresponds to the right and left polarisations
respectively. We can further simplify for small angles of wave
propogation, so $\cos(\theta)\approx 1$ and $v \approx 0$.
\begin{align*} n_{\pm}^2 &\approx S \pm D \cos(\theta) \\
& \approx 1 + r_{\rm p} \pm r_{\rm p} \frac{\omega_{\rm ce}}{\omega} \cos(\theta) \\
& \approx \frac{\omega_{\rm ce}^2 - \omega^2 + \omega_{\rm p}^2}{\omega_{\rm ce}^2 - \omega^2} \pm \frac{\omega_{\rm p}^2}{\omega_{\rm ce}^2 -\omega^2} \frac{\omega_{\rm ce}}{\omega} \cos(\theta) \\
& \approx 1  \mp \frac{\omega_{\rm p}^2 \omega_{\rm ce}}{\omega^3} \cos(\theta)
\end{align*}
For $\omega^2 \gg \omega_{\rm p}^2$, $\omega^2 \gg \omega_{\rm ce}^2$.
Then
\begin{align*}
  n_+ & \approx \qty( 1 - \frac{\omega_{\rm p}^2 \omega_{\rm ce}}{\omega^3} \cos(\theta))^{\half} \approx 1 - \frac{\omega_{\rm p}^2 \omega_{\rm ce}}{2 \omega^3 \cos(\theta)} \\
  n_- & \approx \qty( 1+ \frac{\omega_{\rm p}^2 \omega_{\rm ce}}{\omega^3} \cos(\theta) )^{\half} \approx 1 + \frac{\omega_{\rm p}^2 \omega_{\rm ce}}{2 \omega^3} \cos(\theta)
\end{align*}
And so,
\[ \Delta n = n_- - n_+ \approx \frac{\omega_{\rm p}^2 \omega_{\rm ce}}{\omega^3} \cos(\theta) \]
Since $n = \frac{kc}{\omega}$, we know $k = \frac{\omega}{c} n$, and so
\begin{align*}
  K_- - K_+ \approx \frac{\omega_{\rm p}^2 \omega_{\rm ce}}{c \omega^2} \cos(\theta) 
\end{align*}
and so finally, the rotation angle is
\begin{equation}
  \label{eq:13}
  \dd{\phi} = \half \frac{\omega_{\rm p}^2 \omega_{\rm ce}}{c \omega^2} \cos(\theta) \dd{z}
\end{equation}
At the high wave frequency limit,
\[ \omega = k c = \frac{2 \pi c}{\lambda} \] Hence, as $\omega_{\rm p}
= \sqrt{\frac{4 \pi e^2 n_{\rm e}}{4 \pi m \epsilon_0}}$, and
$\omega_{\rm c} = \frac{eB}{m}$,
\begin{align*}
  \dv{\phi}{z} &=  \half \frac{e^2 n }{m \epsilon_0} \frac{eB}{m} \frac{1 \lambda^2}{c (2 \pi)^2 c^2} \cos(\theta) \\
&= \frac{e^3 B n_{\rm e} \lambda^2}{8 \pi^2 m^2 \epsilon_0 c^3} \cos(\theta)
\end{align*}
And then, to find the Farday angle after the radiation has travelled
along a path $r$,
\begin{equation}
  \label{eq:14}
  \phi_{\rm F} = \frac{e^3  \lambda^2}{8 \pi^2 m^2 \epsilon_0 c^3} \int_0^r B(z) n_{\rm e}(z) \cos(\theta) \dd{z}
\end{equation}
Assuming a uniform number density the Faraday rotation is a measure of
the magnetic field along the path.\\
Faraday rotation occurs because light which is polarised in the
direction of electron gyration has a higher phase speed than that
which is polarised in the opposing direction. The rotation is higher
for low-frequency waves.



  \nocite{*} \bibliographystyle{plain}
\bibliography{plasma}

\end{document}